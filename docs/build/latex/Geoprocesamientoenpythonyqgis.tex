%% Generated by Sphinx.
\def\sphinxdocclass{report}
\documentclass[letterpaper,10pt,spanish]{sphinxmanual}
\ifdefined\pdfpxdimen
   \let\sphinxpxdimen\pdfpxdimen\else\newdimen\sphinxpxdimen
\fi \sphinxpxdimen=.75bp\relax

\PassOptionsToPackage{warn}{textcomp}
\usepackage[utf8]{inputenc}
\ifdefined\DeclareUnicodeCharacter
% support both utf8 and utf8x syntaxes
\edef\sphinxdqmaybe{\ifdefined\DeclareUnicodeCharacterAsOptional\string"\fi}
  \DeclareUnicodeCharacter{\sphinxdqmaybe00A0}{\nobreakspace}
  \DeclareUnicodeCharacter{\sphinxdqmaybe2500}{\sphinxunichar{2500}}
  \DeclareUnicodeCharacter{\sphinxdqmaybe2502}{\sphinxunichar{2502}}
  \DeclareUnicodeCharacter{\sphinxdqmaybe2514}{\sphinxunichar{2514}}
  \DeclareUnicodeCharacter{\sphinxdqmaybe251C}{\sphinxunichar{251C}}
  \DeclareUnicodeCharacter{\sphinxdqmaybe2572}{\textbackslash}
\fi
\usepackage{cmap}
\usepackage[T1]{fontenc}
\usepackage{amsmath,amssymb,amstext}
\usepackage{babel}
\usepackage{times}
\usepackage[Sonny]{fncychap}
\ChNameVar{\Large\normalfont\sffamily}
\ChTitleVar{\Large\normalfont\sffamily}
\usepackage{sphinx}

\fvset{fontsize=\small}
\usepackage{geometry}

% Include hyperref last.
\usepackage{hyperref}
% Fix anchor placement for figures with captions.
\usepackage{hypcap}% it must be loaded after hyperref.
% Set up styles of URL: it should be placed after hyperref.
\urlstyle{same}
\addto\captionsspanish{\renewcommand{\contentsname}{Guias de uso:}}

\addto\captionsspanish{\renewcommand{\figurename}{Figura }}
\makeatletter
\def\fnum@figure{\figurename\thefigure{}}
\makeatother
\addto\captionsspanish{\renewcommand{\tablename}{Tabla }}
\makeatletter
\def\fnum@table{\tablename\thetable{}}
\makeatother
\addto\captionsspanish{\renewcommand{\literalblockname}{Lista}}

\addto\captionsspanish{\renewcommand{\literalblockcontinuedname}{proviene de la página anterior}}
\addto\captionsspanish{\renewcommand{\literalblockcontinuesname}{continué en la próxima página}}
\addto\captionsspanish{\renewcommand{\sphinxnonalphabeticalgroupname}{Non-alphabetical}}
\addto\captionsspanish{\renewcommand{\sphinxsymbolsname}{Símbolos}}
\addto\captionsspanish{\renewcommand{\sphinxnumbersname}{Numbers}}

\addto\extrasspanish{\def\pageautorefname{página}}

\setcounter{tocdepth}{0}



\title{Geoprocesamiento en python y qgis}
\date{19 de marzo de 2021}
\release{1.0}
\author{Víctor Hernández}
\newcommand{\sphinxlogo}{\vbox{}}
\renewcommand{\releasename}{Versión}
\makeindex
\begin{document}

\ifdefined\shorthandoff
  \ifnum\catcode`\=\string=\active\shorthandoff{=}\fi
  \ifnum\catcode`\"=\active\shorthandoff{"}\fi
\fi

\pagestyle{empty}
\sphinxmaketitle
\pagestyle{plain}
\sphinxtableofcontents
\pagestyle{normal}
\phantomsection\label{\detokenize{index::doc}}



\chapter{¿Cómo ejecutar un código en Qgis?}
\label{\detokenize{ejecucion:como-ejecutar-un-codigo-en-qgis}}\label{\detokenize{ejecucion::doc}}

\section{Paso \#1}
\label{\detokenize{ejecucion:paso-1}}
Ejecutar Qgis Desktop 3.XX, la ventana que se muestra es la que
corresponde a la interfaz gráfica del programa, en la barra de tareas
hacer clic en el ícono correspondiente a \sphinxstylestrong{python} para abrir la consola

\noindent\sphinxincludegraphics{{qgis_interfaz}.png}


\section{paso \#2}
\label{\detokenize{ejecucion:paso-2}}
En la parte inferior de la ventana se mostrará la consola de python

\noindent\sphinxincludegraphics{{consola_pyqgis}.png}


\section{paso \#3}
\label{\detokenize{ejecucion:paso-3}}
Hacer clic en el ícono de \sphinxstylestrong{Editor}

\noindent\sphinxincludegraphics{{consola_editor}.png}


\section{paso \#4}
\label{\detokenize{ejecucion:paso-4}}
Se despliega en el lado izquierdo un panel, el cual es el editor
de código, cuenta con una barra de tareas, para abrir un script
dar clic en el ícono de \sphinxstylestrong{abrir archivo}

\noindent\sphinxincludegraphics{{consola_editor_interfaz}.png}


\section{paso \#5}
\label{\detokenize{ejecucion:paso-5}}
Se abrirá una ventana que te permite usar el explorador de archivos
para navegar y encontrar el archivo \sphinxstylestrong{.py}, elegir el script deseado y
dar clic en abrir.

\noindent\sphinxincludegraphics{{abrir_script}.PNG}


\chapter{Libreria de funciones para el análsis espacial multicriterio}
\label{\detokenize{apcsig:libreria-de-funciones-para-el-analsis-espacial-multicriterio}}\label{\detokenize{apcsig::doc}}
En esta página encontrarás la documentación de las funciones que he creado en python y qgis 3.10 o superior


\section{Requerimientos generales}
\label{\detokenize{apcsig:requerimientos-generales}}
Para asegurar la ejecución correcta del código es importante
verificar la instalación y funcionamiento de los siguientes elementos:
\begin{itemize}
\item {} 
Qgis 3.10 o superior con Grass y librerías de Osgeo4W

\item {} 
Librerías python:

\end{itemize}
\begin{itemize}
\item {} 
Numpy

\item {} 
Pandas

\item {} 
GDAL

\item {} 
reduce

\item {} 
os

\item {} 
copy

\item {} 
pprint

\item {} 
string

\end{itemize}


\section{Descarga la libreria}
\label{\detokenize{apcsig:descarga-la-libreria}}
\sphinxcode{\sphinxupquote{apcsig.py}}.


\section{Documentación}
\label{\detokenize{apcsig:module-apcsig}}\label{\detokenize{apcsig:documentacion}}\index{apcsig (módulo)@\spxentry{apcsig}\spxextra{módulo}}
Autor: Víctor Hernández D.

Correo: \sphinxhref{mailto:victor.hernandez@iecologia.unam.mx}{victor.hernandez@iecologia.unam.mx}

user\_github: @vichdzgeo

Colaboladores: LANCIS - UNAM
\index{agregar\_categorias() (en el módulo apcsig)@\spxentry{agregar\_categorias()}\spxextra{en el módulo apcsig}}

\begin{fulllineitems}
\phantomsection\label{\detokenize{apcsig:apcsig.agregar_categorias}}\pysiglinewithargsret{\sphinxcode{\sphinxupquote{apcsig.}}\sphinxbfcode{\sphinxupquote{agregar\_categorias}}}{\emph{path\_v}, \emph{campo}, \emph{nuevo\_int\_cats='categorias'}, \emph{cont=1}}{}
Esta función reclasifica una capa en enteros consecutivos en función de las categorias únicas de un 
campo en especificico, como subproducto genera un archivo csv con las categorias.
\begin{quote}\begin{description}
\item[{Parámetros}] \leavevmode\begin{itemize}
\item {} 
\sphinxstyleliteralstrong{\sphinxupquote{path\_v}} (\sphinxhref{https://docs.python.org/3/library/stdtypes.html\#str}{\sphinxstyleliteralemphasis{\sphinxupquote{str}}}) \textendash{} ruta de capa vectorial

\item {} 
\sphinxstyleliteralstrong{\sphinxupquote{campo}} (\sphinxhref{https://docs.python.org/3/library/stdtypes.html\#str}{\sphinxstyleliteralemphasis{\sphinxupquote{str}}}) \textendash{} nombre del campo que contiene las categorias

\item {} 
\sphinxstyleliteralstrong{\sphinxupquote{nuevo\_int\_cats}} (\sphinxhref{https://docs.python.org/3/library/stdtypes.html\#str}{\sphinxstyleliteralemphasis{\sphinxupquote{str}}}) \textendash{} nombre del campo a crear, el cúal contendrá las categorias en enteros

\item {} 
\sphinxstyleliteralstrong{\sphinxupquote{cont}} (\sphinxhref{https://docs.python.org/3/library/functions.html\#int}{\sphinxstyleliteralemphasis{\sphinxupquote{int}}}) \textendash{} contador para empezar la numeración en el número indicado, por defecto es 1

\end{itemize}

\end{description}\end{quote}

\end{fulllineitems}

\index{alinear\_raster() (en el módulo apcsig)@\spxentry{alinear\_raster()}\spxextra{en el módulo apcsig}}

\begin{fulllineitems}
\phantomsection\label{\detokenize{apcsig:apcsig.alinear_raster}}\pysiglinewithargsret{\sphinxcode{\sphinxupquote{apcsig.}}\sphinxbfcode{\sphinxupquote{alinear\_raster}}}{\emph{path\_raster}, \emph{region}, \emph{resolucion}, \emph{path\_salida}, \emph{crs\_destino=''}, \emph{tipo='int'}}{}
Esta función alinea un raster dada una región y el tamaño de pixel 
:param path\_raster: ruta de la capa a alinear
:type path\_raster: str
\begin{quote}\begin{description}
\item[{Parámetros}] \leavevmode\begin{itemize}
\item {} 
\sphinxstyleliteralstrong{\sphinxupquote{region}} (\sphinxhref{https://docs.python.org/3/library/stdtypes.html\#str}{\sphinxstyleliteralemphasis{\sphinxupquote{str}}}) \textendash{} coordenadas de la región del estudio  xmin,xmax,ymin,ymax

\item {} 
\sphinxstyleliteralstrong{\sphinxupquote{resolucion}} (\sphinxhref{https://docs.python.org/3/library/functions.html\#int}{\sphinxstyleliteralemphasis{\sphinxupquote{int}}}) \textendash{} tamaño de pixel

\item {} 
\sphinxstyleliteralstrong{\sphinxupquote{path\_salida}} (\sphinxhref{https://docs.python.org/3/library/stdtypes.html\#str}{\sphinxstyleliteralemphasis{\sphinxupquote{str}}}) \textendash{} ruta de la capa de salida con extension tif

\item {} 
\sphinxstyleliteralstrong{\sphinxupquote{crs\_destino}} (\sphinxhref{https://docs.python.org/3/library/stdtypes.html\#str}{\sphinxstyleliteralemphasis{\sphinxupquote{str}}}) \textendash{} nombre del código EPSG,

\item {} 
\sphinxstyleliteralstrong{\sphinxupquote{tipo}} (\sphinxhref{https://docs.python.org/3/library/stdtypes.html\#str}{\sphinxstyleliteralemphasis{\sphinxupquote{str}}}) \textendash{} tipo de dato, use “int” para entero o “float” para flotante, por default es entero

\end{itemize}

\end{description}\end{quote}

\end{fulllineitems}

\index{aplica\_mascara() (en el módulo apcsig)@\spxentry{aplica\_mascara()}\spxextra{en el módulo apcsig}}

\begin{fulllineitems}
\phantomsection\label{\detokenize{apcsig:apcsig.aplica_mascara}}\pysiglinewithargsret{\sphinxcode{\sphinxupquote{apcsig.}}\sphinxbfcode{\sphinxupquote{aplica\_mascara}}}{\emph{path\_mascara}, \emph{path\_capa}, \emph{path\_salida}, \emph{region}}{}
Esta función aplica la máscara de la zona de estudio a una capa raster, 
es importante que la capa a la cual se aplicará la máscara este previamente alineada
\begin{quote}\begin{description}
\item[{Parámetros}] \leavevmode\begin{itemize}
\item {} 
\sphinxstyleliteralstrong{\sphinxupquote{path\_mascara}} (\sphinxhref{https://docs.python.org/3/library/stdtypes.html\#str}{\sphinxstyleliteralemphasis{\sphinxupquote{str}}}) \textendash{} ruta de la mascara en formato tiff

\item {} 
\sphinxstyleliteralstrong{\sphinxupquote{path\_capa}} (\sphinxhref{https://docs.python.org/3/library/stdtypes.html\#str}{\sphinxstyleliteralemphasis{\sphinxupquote{str}}}) \textendash{} ruta de la capa a la cual se requiere aplicar la máscara

\item {} 
\sphinxstyleliteralstrong{\sphinxupquote{path\_salida}} (\sphinxhref{https://docs.python.org/3/library/stdtypes.html\#str}{\sphinxstyleliteralemphasis{\sphinxupquote{str}}}) \textendash{} ruta de la capa resultado de aplicar la máscara

\item {} 
\sphinxstyleliteralstrong{\sphinxupquote{region}} (\sphinxhref{https://docs.python.org/3/library/stdtypes.html\#str}{\sphinxstyleliteralemphasis{\sphinxupquote{str}}}) \textendash{} coordenadas de la región del estudio  xmin,xmax,ymin,ymax

\end{itemize}

\end{description}\end{quote}

\end{fulllineitems}

\index{asignar\_nulls() (en el módulo apcsig)@\spxentry{asignar\_nulls()}\spxextra{en el módulo apcsig}}

\begin{fulllineitems}
\phantomsection\label{\detokenize{apcsig:apcsig.asignar_nulls}}\pysiglinewithargsret{\sphinxcode{\sphinxupquote{apcsig.}}\sphinxbfcode{\sphinxupquote{asignar\_nulls}}}{\emph{map}, \emph{output}, \emph{valor\_huecos=0}}{}
Esta función asigna un valor  a los no\_data de la capa
\begin{quote}\begin{description}
\item[{Parámetros}] \leavevmode\begin{itemize}
\item {} 
\sphinxstyleliteralstrong{\sphinxupquote{map}} (\sphinxhref{https://docs.python.org/3/library/stdtypes.html\#str}{\sphinxstyleliteralemphasis{\sphinxupquote{str}}}) \textendash{} ruta de la capa raster

\item {} 
\sphinxstyleliteralstrong{\sphinxupquote{region}} (\sphinxhref{https://docs.python.org/3/library/stdtypes.html\#str}{\sphinxstyleliteralemphasis{\sphinxupquote{str}}}) \textendash{} coordenadas de la región del estudio  xmin,xmax,ymin,ymax

\end{itemize}

\end{description}\end{quote}

:param output:ruta de la capa resultante
:type output: str
\begin{quote}\begin{description}
\item[{Parámetros}] \leavevmode
\sphinxstyleliteralstrong{\sphinxupquote{valor\_huecos}} (\sphinxhref{https://docs.python.org/3/library/functions.html\#int}{\sphinxstyleliteralemphasis{\sphinxupquote{int}}}) \textendash{} número que tendrán los pixeles nulos

\end{description}\end{quote}

\end{fulllineitems}

\index{calculadora\_grass() (en el módulo apcsig)@\spxentry{calculadora\_grass()}\spxextra{en el módulo apcsig}}

\begin{fulllineitems}
\phantomsection\label{\detokenize{apcsig:apcsig.calculadora_grass}}\pysiglinewithargsret{\sphinxcode{\sphinxupquote{apcsig.}}\sphinxbfcode{\sphinxupquote{calculadora\_grass}}}{\emph{path\_capa}, \emph{ecuacion}, \emph{path\_salida}}{}
Esta función aplica la máscara de la zona de estudio
\begin{quote}\begin{description}
\item[{Parámetros}] \leavevmode\begin{itemize}
\item {} 
\sphinxstyleliteralstrong{\sphinxupquote{path\_mascara}} (\sphinxhref{https://docs.python.org/3/library/stdtypes.html\#str}{\sphinxstyleliteralemphasis{\sphinxupquote{str}}}) \textendash{} ruta de la mascara en formato tiff

\item {} 
\sphinxstyleliteralstrong{\sphinxupquote{path\_capa}} (\sphinxhref{https://docs.python.org/3/library/stdtypes.html\#str}{\sphinxstyleliteralemphasis{\sphinxupquote{str}}}) \textendash{} ruta de la capa a la cual se requiere aplicar la máscara

\item {} 
\sphinxstyleliteralstrong{\sphinxupquote{path\_salida}} (\sphinxhref{https://docs.python.org/3/library/stdtypes.html\#str}{\sphinxstyleliteralemphasis{\sphinxupquote{str}}}) \textendash{} ruta de la capa resultado de aplicar la máscara

\item {} 
\sphinxstyleliteralstrong{\sphinxupquote{region}} (\sphinxhref{https://docs.python.org/3/library/stdtypes.html\#str}{\sphinxstyleliteralemphasis{\sphinxupquote{str}}}) \textendash{} coordenadas de la región del estudio  xmin,xmax,ymin,ymax

\end{itemize}

\end{description}\end{quote}

\end{fulllineitems}

\index{calculadora\_grass\_2capas() (en el módulo apcsig)@\spxentry{calculadora\_grass\_2capas()}\spxextra{en el módulo apcsig}}

\begin{fulllineitems}
\phantomsection\label{\detokenize{apcsig:apcsig.calculadora_grass_2capas}}\pysiglinewithargsret{\sphinxcode{\sphinxupquote{apcsig.}}\sphinxbfcode{\sphinxupquote{calculadora\_grass\_2capas}}}{\emph{path\_capa\_a}, \emph{path\_capa\_b}, \emph{ecuacion}, \emph{path\_salida}}{}
Esta función aplica la máscara de la zona de estudio
\begin{quote}\begin{description}
\item[{Parámetros}] \leavevmode\begin{itemize}
\item {} 
\sphinxstyleliteralstrong{\sphinxupquote{path\_mascara}} (\sphinxhref{https://docs.python.org/3/library/stdtypes.html\#str}{\sphinxstyleliteralemphasis{\sphinxupquote{str}}}) \textendash{} ruta de la mascara en formato tiff

\item {} 
\sphinxstyleliteralstrong{\sphinxupquote{path\_capa}} (\sphinxhref{https://docs.python.org/3/library/stdtypes.html\#str}{\sphinxstyleliteralemphasis{\sphinxupquote{str}}}) \textendash{} ruta de la capa a la cual se requiere aplicar la máscara

\item {} 
\sphinxstyleliteralstrong{\sphinxupquote{path\_salida}} (\sphinxhref{https://docs.python.org/3/library/stdtypes.html\#str}{\sphinxstyleliteralemphasis{\sphinxupquote{str}}}) \textendash{} ruta de la capa resultado de aplicar la máscara

\item {} 
\sphinxstyleliteralstrong{\sphinxupquote{region}} (\sphinxhref{https://docs.python.org/3/library/stdtypes.html\#str}{\sphinxstyleliteralemphasis{\sphinxupquote{str}}}) \textendash{} coordenadas de la región del estudio  xmin,xmax,ymin,ymax

\end{itemize}

\end{description}\end{quote}

\end{fulllineitems}

\index{calculadora\_grass\_3capas() (en el módulo apcsig)@\spxentry{calculadora\_grass\_3capas()}\spxextra{en el módulo apcsig}}

\begin{fulllineitems}
\phantomsection\label{\detokenize{apcsig:apcsig.calculadora_grass_3capas}}\pysiglinewithargsret{\sphinxcode{\sphinxupquote{apcsig.}}\sphinxbfcode{\sphinxupquote{calculadora\_grass\_3capas}}}{\emph{path\_capa\_a}, \emph{path\_capa\_b}, \emph{path\_capa\_c}, \emph{ecuacion}, \emph{path\_salida}}{}
Esta función aplica la máscara de la zona de estudio
\begin{quote}\begin{description}
\item[{Parámetros}] \leavevmode\begin{itemize}
\item {} 
\sphinxstyleliteralstrong{\sphinxupquote{path\_mascara}} (\sphinxhref{https://docs.python.org/3/library/stdtypes.html\#str}{\sphinxstyleliteralemphasis{\sphinxupquote{str}}}) \textendash{} ruta de la mascara en formato tiff

\item {} 
\sphinxstyleliteralstrong{\sphinxupquote{path\_capa}} (\sphinxhref{https://docs.python.org/3/library/stdtypes.html\#str}{\sphinxstyleliteralemphasis{\sphinxupquote{str}}}) \textendash{} ruta de la capa a la cual se requiere aplicar la máscara

\item {} 
\sphinxstyleliteralstrong{\sphinxupquote{path\_salida}} (\sphinxhref{https://docs.python.org/3/library/stdtypes.html\#str}{\sphinxstyleliteralemphasis{\sphinxupquote{str}}}) \textendash{} ruta de la capa resultado de aplicar la máscara

\item {} 
\sphinxstyleliteralstrong{\sphinxupquote{region}} (\sphinxhref{https://docs.python.org/3/library/stdtypes.html\#str}{\sphinxstyleliteralemphasis{\sphinxupquote{str}}}) \textendash{} coordenadas de la región del estudio  xmin,xmax,ymin,ymax

\end{itemize}

\end{description}\end{quote}

\end{fulllineitems}

\index{campos\_mayusculas() (en el módulo apcsig)@\spxentry{campos\_mayusculas()}\spxextra{en el módulo apcsig}}

\begin{fulllineitems}
\phantomsection\label{\detokenize{apcsig:apcsig.campos_mayusculas}}\pysiglinewithargsret{\sphinxcode{\sphinxupquote{apcsig.}}\sphinxbfcode{\sphinxupquote{campos\_mayusculas}}}{\emph{path\_shape}}{}
Esta función renombra los campos en mayusculas
\begin{quote}\begin{description}
\item[{Parámetros}] \leavevmode
\sphinxstyleliteralstrong{\sphinxupquote{path\_shape}} (\sphinxhref{https://docs.python.org/3/library/stdtypes.html\#str}{\sphinxstyleliteralemphasis{\sphinxupquote{str}}}) \textendash{} Ruta de la capa vectorial

\end{description}\end{quote}

\end{fulllineitems}

\index{campos\_minusculas() (en el módulo apcsig)@\spxentry{campos\_minusculas()}\spxextra{en el módulo apcsig}}

\begin{fulllineitems}
\phantomsection\label{\detokenize{apcsig:apcsig.campos_minusculas}}\pysiglinewithargsret{\sphinxcode{\sphinxupquote{apcsig.}}\sphinxbfcode{\sphinxupquote{campos\_minusculas}}}{\emph{path\_shape}}{}
Esta función renombra los campos en minúsculas
\begin{quote}\begin{description}
\item[{Parámetros}] \leavevmode
\sphinxstyleliteralstrong{\sphinxupquote{path\_shape}} (\sphinxhref{https://docs.python.org/3/library/stdtypes.html\#str}{\sphinxstyleliteralemphasis{\sphinxupquote{str}}}) \textendash{} Ruta de la capa vectorial

\end{description}\end{quote}

\end{fulllineitems}

\index{capa\_binaria() (en el módulo apcsig)@\spxentry{capa\_binaria()}\spxextra{en el módulo apcsig}}

\begin{fulllineitems}
\phantomsection\label{\detokenize{apcsig:apcsig.capa_binaria}}\pysiglinewithargsret{\sphinxcode{\sphinxupquote{apcsig.}}\sphinxbfcode{\sphinxupquote{capa\_binaria}}}{\emph{path\_v}, \emph{campo\_cat='presencia'}, \emph{valor=1}}{}
Esta función crea un campo llamado presencia y asigna 
a cada elemento el valor de 1
\begin{quote}\begin{description}
\item[{Parámetros}] \leavevmode\begin{itemize}
\item {} 
\sphinxstyleliteralstrong{\sphinxupquote{path\_v}} (\sphinxhref{https://docs.python.org/3/library/stdtypes.html\#str}{\sphinxstyleliteralemphasis{\sphinxupquote{str}}}) \textendash{} ruta de la capa vectorial

\item {} 
\sphinxstyleliteralstrong{\sphinxupquote{campo\_cat}} (\sphinxhref{https://docs.python.org/3/library/stdtypes.html\#str}{\sphinxstyleliteralemphasis{\sphinxupquote{str}}}) \textendash{} nombre del campo a crear, por default es presencia

\end{itemize}

\end{description}\end{quote}

\end{fulllineitems}

\index{categorias\_campo\_csv() (en el módulo apcsig)@\spxentry{categorias\_campo\_csv()}\spxextra{en el módulo apcsig}}

\begin{fulllineitems}
\phantomsection\label{\detokenize{apcsig:apcsig.categorias_campo_csv}}\pysiglinewithargsret{\sphinxcode{\sphinxupquote{apcsig.}}\sphinxbfcode{\sphinxupquote{categorias\_campo\_csv}}}{\emph{path\_shape}, \emph{campo}}{}
Esta función extrae las categorias únicas de un campo dado de una capa vectorial, el 
archivo csv se guarda en la misma ruta de la capa vectorial y es nombrado como : \sphinxstylestrong{categorias\_campo\_nombre\_capa.csv}

\begin{sphinxadmonition}{note}{Nota:}
En dado caso que el nombre de la categoria contenga el símbolo de “,” está función la remplaza por “;” para evitar errores en la escritura del archivo csv
\end{sphinxadmonition}
\begin{quote}\begin{description}
\item[{Parámetros}] \leavevmode\begin{itemize}
\item {} 
\sphinxstyleliteralstrong{\sphinxupquote{path\_shape}} (\sphinxhref{https://docs.python.org/3/library/stdtypes.html\#str}{\sphinxstyleliteralemphasis{\sphinxupquote{str}}}) \textendash{} ruta de la capa vectorial

\item {} 
\sphinxstyleliteralstrong{\sphinxupquote{campo}} (\sphinxhref{https://docs.python.org/3/library/stdtypes.html\#str}{\sphinxstyleliteralemphasis{\sphinxupquote{str}}}) \textendash{} nombre del campo que contiene las categorias

\end{itemize}

\end{description}\end{quote}

\end{fulllineitems}

\index{clasificar\_shape() (en el módulo apcsig)@\spxentry{clasificar\_shape()}\spxextra{en el módulo apcsig}}

\begin{fulllineitems}
\phantomsection\label{\detokenize{apcsig:apcsig.clasificar_shape}}\pysiglinewithargsret{\sphinxcode{\sphinxupquote{apcsig.}}\sphinxbfcode{\sphinxupquote{clasificar\_shape}}}{\emph{path\_v}, \emph{clasificador}, \emph{l\_field}, \emph{campo\_cat=''}, \emph{fp=2}, \emph{categories=5}}{}
Funcion integradora para clasificar la capa vectorial
\begin{quote}\begin{description}
\item[{Parámetros}] \leavevmode\begin{itemize}
\item {} 
\sphinxstyleliteralstrong{\sphinxupquote{path\_v}} (\sphinxhref{https://docs.python.org/3/library/stdtypes.html\#str}{\sphinxstyleliteralemphasis{\sphinxupquote{str}}}) \textendash{} ruta de la capa vectorial

\item {} 
\sphinxstyleliteralstrong{\sphinxupquote{clasificador}} (\sphinxhref{https://docs.python.org/3/library/stdtypes.html\#str}{\sphinxstyleliteralemphasis{\sphinxupquote{str}}}) \textendash{} nombre del clasificador

\item {} 
\sphinxstyleliteralstrong{\sphinxupquote{fp}} (\sphinxhref{https://docs.python.org/3/library/functions.html\#float}{\sphinxstyleliteralemphasis{\sphinxupquote{float}}}) \textendash{} factor de progresión

\item {} 
\sphinxstyleliteralstrong{\sphinxupquote{categories}} (\sphinxhref{https://docs.python.org/3/library/functions.html\#int}{\sphinxstyleliteralemphasis{\sphinxupquote{int}}}) \textendash{} número de categorias

\end{itemize}

\end{description}\end{quote}

\end{fulllineitems}

\index{crea\_capa\_raster() (en el módulo apcsig)@\spxentry{crea\_capa\_raster()}\spxextra{en el módulo apcsig}}

\begin{fulllineitems}
\phantomsection\label{\detokenize{apcsig:apcsig.crea_capa_raster}}\pysiglinewithargsret{\sphinxcode{\sphinxupquote{apcsig.}}\sphinxbfcode{\sphinxupquote{crea\_capa\_raster}}}{\emph{ecuacion}, \emph{rasters\_input}, \emph{salida}, \emph{decimales=3}}{}
Esta función crea una capa mediante la calculadora raster
de GDAL, esta función esta limitada hasta 14 variables en la ecuación.
\begin{quote}\begin{description}
\item[{Parámetros}] \leavevmode\begin{itemize}
\item {} 
\sphinxstyleliteralstrong{\sphinxupquote{ecuacion}} (\sphinxhref{https://docs.python.org/3/library/stdtypes.html\#str}{\sphinxstyleliteralemphasis{\sphinxupquote{str}}}) \textendash{} ecuación expresada en formato gdal,

\item {} 
\sphinxstyleliteralstrong{\sphinxupquote{rasters\_input}} (\sphinxhref{https://docs.python.org/3/library/stdtypes.html\#list}{\sphinxstyleliteralemphasis{\sphinxupquote{list}}}) \textendash{} lista de los paths de los archivos rasters

\item {} 
\sphinxstyleliteralstrong{\sphinxupquote{salida}} (\sphinxhref{https://docs.python.org/3/library/stdtypes.html\#str}{\sphinxstyleliteralemphasis{\sphinxupquote{str}}}) \textendash{} ruta con extensión tiff de la salida

\end{itemize}

\item[{Devuelve}] \leavevmode
Capa raster de tipo flotante, los valores de la capa son redondeados a 3 decimales

\end{description}\end{quote}

\end{fulllineitems}

\index{crear\_campo() (en el módulo apcsig)@\spxentry{crear\_campo()}\spxextra{en el módulo apcsig}}

\begin{fulllineitems}
\phantomsection\label{\detokenize{apcsig:apcsig.crear_campo}}\pysiglinewithargsret{\sphinxcode{\sphinxupquote{apcsig.}}\sphinxbfcode{\sphinxupquote{crear\_campo}}}{\emph{path\_vector}, \emph{nombre\_campo}, \emph{tipo}}{}
Esta funcion crea un campo segun el tipo especificado.
Parametros:
:param path\_vector: La ruta del archivo shapefile al cual se le quiere                         agregar el campo
:type path\_vector: String
\begin{quote}\begin{description}
\item[{Parámetros}] \leavevmode\begin{itemize}
\item {} 
\sphinxstyleliteralstrong{\sphinxupquote{nombre\_campo}} (\sphinxstyleliteralemphasis{\sphinxupquote{Sting}}) \textendash{} Nombre del campo nuevo

\item {} 
\sphinxstyleliteralstrong{\sphinxupquote{tipo}} \textendash{} es el tipo de campo que se quiere crear

\end{itemize}

\end{description}\end{quote}

Int: para crear un campo tipo entero
Double: para crear un campo tipo doble o flotante
String: para crear un campo tipo texto
Date: para crear un campo tipo fecha
:type tipo: String

\end{fulllineitems}

\index{cuantiles\_s() (en el módulo apcsig)@\spxentry{cuantiles\_s()}\spxextra{en el módulo apcsig}}

\begin{fulllineitems}
\phantomsection\label{\detokenize{apcsig:apcsig.cuantiles_s}}\pysiglinewithargsret{\sphinxcode{\sphinxupquote{apcsig.}}\sphinxbfcode{\sphinxupquote{cuantiles\_s}}}{\emph{path\_v}, \emph{quantil}, \emph{field}, \emph{min}, \emph{max}}{}
Esta función regresa la lista de cortes según el cualtil 
deseado de los valores de un campo de la capa vectorial de entrada
\begin{quote}\begin{description}
\item[{Parámetros}] \leavevmode\begin{itemize}
\item {} 
\sphinxstyleliteralstrong{\sphinxupquote{path\_v}} (\sphinxhref{https://docs.python.org/3/library/stdtypes.html\#str}{\sphinxstyleliteralemphasis{\sphinxupquote{str}}}) \textendash{} ruta de la capa vectorial

\item {} 
\sphinxstyleliteralstrong{\sphinxupquote{quantil}} (\sphinxhref{https://docs.python.org/3/library/functions.html\#int}{\sphinxstyleliteralemphasis{\sphinxupquote{int}}}) \textendash{} cuantil

\item {} 
\sphinxstyleliteralstrong{\sphinxupquote{field}} (\sphinxhref{https://docs.python.org/3/library/stdtypes.html\#str}{\sphinxstyleliteralemphasis{\sphinxupquote{str}}}) \textendash{} nombre del campo

\item {} 
\sphinxstyleliteralstrong{\sphinxupquote{min}} (\sphinxhref{https://docs.python.org/3/library/functions.html\#float}{\sphinxstyleliteralemphasis{\sphinxupquote{float}}}) \textendash{} valor mínimo de la capa

\item {} 
\sphinxstyleliteralstrong{\sphinxupquote{max}} (\sphinxhref{https://docs.python.org/3/library/functions.html\#float}{\sphinxstyleliteralemphasis{\sphinxupquote{float}}}) \textendash{} valor máximo de la capa

\end{itemize}

\end{description}\end{quote}

\end{fulllineitems}

\index{distancia\_caminos\_lugar() (en el módulo apcsig)@\spxentry{distancia\_caminos\_lugar()}\spxextra{en el módulo apcsig}}

\begin{fulllineitems}
\phantomsection\label{\detokenize{apcsig:apcsig.distancia_caminos_lugar}}\pysiglinewithargsret{\sphinxcode{\sphinxupquote{apcsig.}}\sphinxbfcode{\sphinxupquote{distancia\_caminos\_lugar}}}{\emph{layer\_raster\_lugar}, \emph{path\_caminos}, \emph{campo\_caminos}, \emph{path\_mascara}, \emph{path\_salida}, \emph{region\_ext}, \emph{ancho\_ext=2292}, \emph{alto\_ext=2284}, \emph{remover=1}}{}
Esta función genera una capa de distancia a caminos y agrega distancia cero a aquellos pixeles que se sobreponen con el área del lugar

\end{fulllineitems}

\index{ecuacion\_clp() (en el módulo apcsig)@\spxentry{ecuacion\_clp()}\spxextra{en el módulo apcsig}}

\begin{fulllineitems}
\phantomsection\label{\detokenize{apcsig:apcsig.ecuacion_clp}}\pysiglinewithargsret{\sphinxcode{\sphinxupquote{apcsig.}}\sphinxbfcode{\sphinxupquote{ecuacion\_clp}}}{\emph{pesos}}{}
Esta función recibe una lista de pesos para regresar la ecuación
en la estructura requerida por gdal para la combinación lineal ponderada.

\sphinxstylestrong{ejemplo:}
\begin{equation*}
\begin{split}ecuacion = A*0.40 + B*0.25 + C*0.15 +D*0.20\end{split}
\end{equation*}\begin{quote}\begin{description}
\item[{Parámetros}] \leavevmode
\sphinxstyleliteralstrong{\sphinxupquote{pesos}} (\sphinxstyleliteralemphasis{\sphinxupquote{lista}}) \textendash{} lista de los pesos de las capas, salida de la función

\item[{Devuelve}] \leavevmode
ecuación en formato gdal para ser ingresada a la funcion crea\_capa\_raster

\end{description}\end{quote}

\end{fulllineitems}

\index{equidistantes() (en el módulo apcsig)@\spxentry{equidistantes()}\spxextra{en el módulo apcsig}}

\begin{fulllineitems}
\phantomsection\label{\detokenize{apcsig:apcsig.equidistantes}}\pysiglinewithargsret{\sphinxcode{\sphinxupquote{apcsig.}}\sphinxbfcode{\sphinxupquote{equidistantes}}}{\emph{categories=5}, \emph{min=0}, \emph{max=1}}{}
Esta función regresa la lista de cortes equidistantes según el número 
de categorias y el valor minimo y maximo ingresados.
\begin{quote}\begin{description}
\item[{Parámetros}] \leavevmode\begin{itemize}
\item {} 
\sphinxstyleliteralstrong{\sphinxupquote{categories}} (\sphinxhref{https://docs.python.org/3/library/functions.html\#int}{\sphinxstyleliteralemphasis{\sphinxupquote{int}}}) \textendash{} número de categorias

\item {} 
\sphinxstyleliteralstrong{\sphinxupquote{min}} (\sphinxhref{https://docs.python.org/3/library/functions.html\#float}{\sphinxstyleliteralemphasis{\sphinxupquote{float}}}) \textendash{} valor mínimo de la capa

\item {} 
\sphinxstyleliteralstrong{\sphinxupquote{max}} (\sphinxhref{https://docs.python.org/3/library/functions.html\#float}{\sphinxstyleliteralemphasis{\sphinxupquote{float}}}) \textendash{} valor máximo de la capa

\end{itemize}

\end{description}\end{quote}

\end{fulllineitems}

\index{get\_region() (en el módulo apcsig)@\spxentry{get\_region()}\spxextra{en el módulo apcsig}}

\begin{fulllineitems}
\phantomsection\label{\detokenize{apcsig:apcsig.get_region}}\pysiglinewithargsret{\sphinxcode{\sphinxupquote{apcsig.}}\sphinxbfcode{\sphinxupquote{get\_region}}}{\emph{path\_layer}}{}
Esta función extrae la región o extensión de una capa raster así como el número de columnas y renglones
\begin{quote}\begin{description}
\item[{Parámetros}] \leavevmode
\sphinxstyleliteralstrong{\sphinxupquote{path\_layer}} (\sphinxhref{https://docs.python.org/3/library/stdtypes.html\#str}{\sphinxstyleliteralemphasis{\sphinxupquote{str}}}) \textendash{} ruta de la capa raster

\item[{Devuelve}] \leavevmode
en forma de lista {[}1,2,3{]} {[}1{]} las coordenadas de la extensión de una capa raster xmin,xmax,ymin,ymax ; {[}2{]}. ancho de una capa raster (número de columnas) y {[}3{]}. alto de una capa raster (número de renglones)

\end{description}\end{quote}

\end{fulllineitems}

\index{integra\_localidades\_caminos() (en el módulo apcsig)@\spxentry{integra\_localidades\_caminos()}\spxextra{en el módulo apcsig}}

\begin{fulllineitems}
\phantomsection\label{\detokenize{apcsig:apcsig.integra_localidades_caminos}}\pysiglinewithargsret{\sphinxcode{\sphinxupquote{apcsig.}}\sphinxbfcode{\sphinxupquote{integra\_localidades\_caminos}}}{\emph{path\_lugar\_n}, \emph{w\_lugar}, \emph{path\_d\_camino\_n}, \emph{w\_d\_camino}, \emph{d\_max\_lugar}, \emph{salida}}{}
Esta función aplica la máscara de la zona de estudio
\begin{quote}\begin{description}
\item[{Parámetros}] \leavevmode\begin{itemize}
\item {} 
\sphinxstyleliteralstrong{\sphinxupquote{path\_mascara}} (\sphinxhref{https://docs.python.org/3/library/stdtypes.html\#str}{\sphinxstyleliteralemphasis{\sphinxupquote{str}}}) \textendash{} ruta de la mascara en formato tiff

\item {} 
\sphinxstyleliteralstrong{\sphinxupquote{path\_capa}} (\sphinxhref{https://docs.python.org/3/library/stdtypes.html\#str}{\sphinxstyleliteralemphasis{\sphinxupquote{str}}}) \textendash{} ruta de la capa a la cual se requiere aplicar la máscara

\item {} 
\sphinxstyleliteralstrong{\sphinxupquote{path\_salida}} (\sphinxhref{https://docs.python.org/3/library/stdtypes.html\#str}{\sphinxstyleliteralemphasis{\sphinxupquote{str}}}) \textendash{} ruta de la capa resultado de aplicar la máscara

\item {} 
\sphinxstyleliteralstrong{\sphinxupquote{region}} (\sphinxhref{https://docs.python.org/3/library/stdtypes.html\#str}{\sphinxstyleliteralemphasis{\sphinxupquote{str}}}) \textendash{} coordenadas de la región del estudio  xmin,xmax,ymin,ymax

\end{itemize}

\end{description}\end{quote}

\end{fulllineitems}

\index{llenar\_campos\_nulos() (en el módulo apcsig)@\spxentry{llenar\_campos\_nulos()}\spxextra{en el módulo apcsig}}

\begin{fulllineitems}
\phantomsection\label{\detokenize{apcsig:apcsig.llenar_campos_nulos}}\pysiglinewithargsret{\sphinxcode{\sphinxupquote{apcsig.}}\sphinxbfcode{\sphinxupquote{llenar\_campos\_nulos}}}{\emph{path\_vector}, \emph{valor=-9999}}{}~\begin{quote}\begin{description}
\item[{Parámetros}] \leavevmode\begin{itemize}
\item {} 
\sphinxstyleliteralstrong{\sphinxupquote{path\_vector}} (\sphinxhref{https://docs.python.org/3/library/stdtypes.html\#str}{\sphinxstyleliteralemphasis{\sphinxupquote{str}}}) \textendash{} Ruta de la capa vectorial

\item {} 
\sphinxstyleliteralstrong{\sphinxupquote{valor}} \textendash{} valor numérico para rellenar los campos vacios por default se establece -9999

\end{itemize}

\end{description}\end{quote}

\end{fulllineitems}

\index{max\_min\_vector() (en el módulo apcsig)@\spxentry{max\_min\_vector()}\spxextra{en el módulo apcsig}}

\begin{fulllineitems}
\phantomsection\label{\detokenize{apcsig:apcsig.max_min_vector}}\pysiglinewithargsret{\sphinxcode{\sphinxupquote{apcsig.}}\sphinxbfcode{\sphinxupquote{max\_min\_vector}}}{\emph{layer}, \emph{campo}}{}
Esta función regresa el maximo y minimo del campo
elegido de la capa vectorial de entrada
\begin{quote}\begin{description}
\item[{Parámetros}] \leavevmode\begin{itemize}
\item {} 
\sphinxstyleliteralstrong{\sphinxupquote{layer}} (\sphinxstyleliteralemphasis{\sphinxupquote{QgsLayer}}) \textendash{} capa vectorial

\item {} 
\sphinxstyleliteralstrong{\sphinxupquote{campo}} (\sphinxhref{https://docs.python.org/3/library/stdtypes.html\#str}{\sphinxstyleliteralemphasis{\sphinxupquote{str}}}) \textendash{} nombre del campo

\end{itemize}

\end{description}\end{quote}

\end{fulllineitems}

\index{normailiza() (en el módulo apcsig)@\spxentry{normailiza()}\spxextra{en el módulo apcsig}}

\begin{fulllineitems}
\phantomsection\label{\detokenize{apcsig:apcsig.normailiza}}\pysiglinewithargsret{\sphinxcode{\sphinxupquote{apcsig.}}\sphinxbfcode{\sphinxupquote{normailiza}}}{\emph{path\_raster}, \emph{path\_raster\_n}, \emph{modo='ideales'}}{}
Esta función normaliza una capa raster, se puede elegir entre dos tipos de normalización.
\begin{enumerate}
\def\theenumi{\arabic{enumi}}
\def\labelenumi{\theenumi )}
\makeatletter\def\p@enumii{\p@enumi \theenumi )}\makeatother
\item {} 
ideales: La capa ráster se divide entre el valor máximo como resultado se tiene una capa con un máximo de 1 pero el valor mínimo no necesariamente será 0

\end{enumerate}
\begin{equation*}
\begin{split}ideales=\frac{A}{A.max}\end{split}
\end{equation*}
2) lineal: la capa ráster resultante tendra como valor máximo 1 y como valor mínimo 0
el 1 representará el maximo de la capa de entrada y el 0 representa el valor mínimo de la 
capa de entrada
\begin{equation*}
\begin{split}lineal=\frac{A -A.min}{A.max - A.min}\end{split}
\end{equation*}
\end{fulllineitems}

\index{nulls() (en el módulo apcsig)@\spxentry{nulls()}\spxextra{en el módulo apcsig}}

\begin{fulllineitems}
\phantomsection\label{\detokenize{apcsig:apcsig.nulls}}\pysiglinewithargsret{\sphinxcode{\sphinxupquote{apcsig.}}\sphinxbfcode{\sphinxupquote{nulls}}}{\emph{map}, \emph{output}, \emph{valor\_huecos=0}}{}
Esta función asigna un valor  a los no\_data de la capa
\begin{quote}\begin{description}
\item[{Parámetros}] \leavevmode\begin{itemize}
\item {} 
\sphinxstyleliteralstrong{\sphinxupquote{map}} (\sphinxhref{https://docs.python.org/3/library/stdtypes.html\#str}{\sphinxstyleliteralemphasis{\sphinxupquote{str}}}) \textendash{} ruta de la capa raster

\item {} 
\sphinxstyleliteralstrong{\sphinxupquote{region}} (\sphinxhref{https://docs.python.org/3/library/stdtypes.html\#str}{\sphinxstyleliteralemphasis{\sphinxupquote{str}}}) \textendash{} coordenadas de la región del estudio  xmin,xmax,ymin,ymax

\end{itemize}

\end{description}\end{quote}

:param output:ruta de la capa resultante
:type output: str
\begin{quote}\begin{description}
\item[{Parámetros}] \leavevmode
\sphinxstyleliteralstrong{\sphinxupquote{valor\_huecos}} (\sphinxhref{https://docs.python.org/3/library/functions.html\#int}{\sphinxstyleliteralemphasis{\sphinxupquote{int}}}) \textendash{} número que tendrán los pixeles nulos

\end{description}\end{quote}

\end{fulllineitems}

\index{raster\_min\_max() (en el módulo apcsig)@\spxentry{raster\_min\_max()}\spxextra{en el módulo apcsig}}

\begin{fulllineitems}
\phantomsection\label{\detokenize{apcsig:apcsig.raster_min_max}}\pysiglinewithargsret{\sphinxcode{\sphinxupquote{apcsig.}}\sphinxbfcode{\sphinxupquote{raster\_min\_max}}}{\emph{path\_raster}}{}
Esta funcion regresa los valores maximos y minimos de una capa raster
\begin{quote}\begin{description}
\item[{Parámetros}] \leavevmode
\sphinxstyleliteralstrong{\sphinxupquote{path\_raster}} (\sphinxhref{https://docs.python.org/3/library/stdtypes.html\#str}{\sphinxstyleliteralemphasis{\sphinxupquote{str}}}) \textendash{} ruta de la capa raster

\end{description}\end{quote}

\end{fulllineitems}

\index{rasterizar\_vector() (en el módulo apcsig)@\spxentry{rasterizar\_vector()}\spxextra{en el módulo apcsig}}

\begin{fulllineitems}
\phantomsection\label{\detokenize{apcsig:apcsig.rasterizar_vector}}\pysiglinewithargsret{\sphinxcode{\sphinxupquote{apcsig.}}\sphinxbfcode{\sphinxupquote{rasterizar\_vector}}}{\emph{path\_vector}, \emph{n\_campo}, \emph{region}, \emph{path\_salida}, \emph{tipo='int'}, \emph{ancho=0}, \emph{alto=0}}{}
Esta función rasteriza una capa vectorial a partir de un campo de tipo numérico y dada una región 
y el numero de columnas (ancho) y el numero de renglones (alto)
\begin{quote}\begin{description}
\item[{Parámetros}] \leavevmode\begin{itemize}
\item {} 
\sphinxstyleliteralstrong{\sphinxupquote{path\_vector}} (\sphinxhref{https://docs.python.org/3/library/stdtypes.html\#str}{\sphinxstyleliteralemphasis{\sphinxupquote{str}}}) \textendash{} ruta de la capa vectorial

\item {} 
\sphinxstyleliteralstrong{\sphinxupquote{n\_campo}} (\sphinxstyleliteralemphasis{\sphinxupquote{srt}}) \textendash{} nombre del campo que contiene los id de las categorias

\item {} 
\sphinxstyleliteralstrong{\sphinxupquote{region}} (\sphinxhref{https://docs.python.org/3/library/stdtypes.html\#str}{\sphinxstyleliteralemphasis{\sphinxupquote{str}}}) \textendash{} coordenadas de la región del estudio  xmin,xmax,ymin,ymax

\item {} 
\sphinxstyleliteralstrong{\sphinxupquote{path\_salida}} (\sphinxhref{https://docs.python.org/3/library/stdtypes.html\#str}{\sphinxstyleliteralemphasis{\sphinxupquote{str}}}) \textendash{} ruta de la capa de salida con extension tif

\item {} 
\sphinxstyleliteralstrong{\sphinxupquote{tipo}} (\sphinxhref{https://docs.python.org/3/library/stdtypes.html\#str}{\sphinxstyleliteralemphasis{\sphinxupquote{str}}}) \textendash{} tipo de dato, use “int” para entero o “float” para flotante, por default es entero

\end{itemize}

\end{description}\end{quote}

\end{fulllineitems}

\index{reclasifica\_capa() (en el módulo apcsig)@\spxentry{reclasifica\_capa()}\spxextra{en el módulo apcsig}}

\begin{fulllineitems}
\phantomsection\label{\detokenize{apcsig:apcsig.reclasifica_capa}}\pysiglinewithargsret{\sphinxcode{\sphinxupquote{apcsig.}}\sphinxbfcode{\sphinxupquote{reclasifica\_capa}}}{\emph{capa}, \emph{region}, \emph{reglas}, \emph{salida}}{}
Esta función permite reclasificar una capa raster y genera un archivo 
xml el cúal contiene el nombre de las categorias.
\begin{quote}\begin{description}
\item[{Parámetros}] \leavevmode\begin{itemize}
\item {} 
\sphinxstyleliteralstrong{\sphinxupquote{capa}} (\sphinxhref{https://docs.python.org/3/library/stdtypes.html\#str}{\sphinxstyleliteralemphasis{\sphinxupquote{str}}}) \textendash{} ruta de capa de entrada

\item {} 
\sphinxstyleliteralstrong{\sphinxupquote{region}} (\sphinxhref{https://docs.python.org/3/library/stdtypes.html\#str}{\sphinxstyleliteralemphasis{\sphinxupquote{str}}}) \textendash{} coordenadas de la región del estudio  xmin,xmax,ymin,ymax

\item {} 
\sphinxstyleliteralstrong{\sphinxupquote{reglas}} (\sphinxhref{https://docs.python.org/3/library/stdtypes.html\#str}{\sphinxstyleliteralemphasis{\sphinxupquote{str}}}) \textendash{} ruta del archivo txt que contiene las reglas de clasificación

\item {} 
\sphinxstyleliteralstrong{\sphinxupquote{salida}} (\sphinxhref{https://docs.python.org/3/library/stdtypes.html\#str}{\sphinxstyleliteralemphasis{\sphinxupquote{str}}}) \textendash{} ruta de la capa de salida reclasificada

\end{itemize}

\item[{Retuns}] \leavevmode
Capa raster clasificada y archivo xml

\end{description}\end{quote}

\end{fulllineitems}

\index{redondea\_raster() (en el módulo apcsig)@\spxentry{redondea\_raster()}\spxextra{en el módulo apcsig}}

\begin{fulllineitems}
\phantomsection\label{\detokenize{apcsig:apcsig.redondea_raster}}\pysiglinewithargsret{\sphinxcode{\sphinxupquote{apcsig.}}\sphinxbfcode{\sphinxupquote{redondea\_raster}}}{\emph{path\_raster}, \emph{salida}, \emph{no\_decimales=3}}{}
Esta función redondea una capa raster de tipo flotante en el número de decimales indicado
\begin{quote}\begin{description}
\item[{Parámetros}] \leavevmode\begin{itemize}
\item {} 
\sphinxstyleliteralstrong{\sphinxupquote{path\_raster}} (\sphinxhref{https://docs.python.org/3/library/stdtypes.html\#str}{\sphinxstyleliteralemphasis{\sphinxupquote{str}}}) \textendash{} ruta de la capa raster

\item {} 
\sphinxstyleliteralstrong{\sphinxupquote{no\_decimales}} \textendash{} número de decimales a los que se va a redondear la capa, por default es 3

\item {} 
\sphinxstyleliteralstrong{\sphinxupquote{salida}} (\sphinxhref{https://docs.python.org/3/library/stdtypes.html\#str}{\sphinxstyleliteralemphasis{\sphinxupquote{str}}}) \textendash{} ruta de la capa de salida

\end{itemize}

\end{description}\end{quote}

\end{fulllineitems}

\index{remove\_raster() (en el módulo apcsig)@\spxentry{remove\_raster()}\spxextra{en el módulo apcsig}}

\begin{fulllineitems}
\phantomsection\label{\detokenize{apcsig:apcsig.remove_raster}}\pysiglinewithargsret{\sphinxcode{\sphinxupquote{apcsig.}}\sphinxbfcode{\sphinxupquote{remove\_raster}}}{\emph{path\_r}}{}
Esta función elimina una capa del sistema
\begin{quote}\begin{description}
\item[{Parámetros}] \leavevmode
\sphinxstyleliteralstrong{\sphinxupquote{path\_r}} (\sphinxhref{https://docs.python.org/3/library/stdtypes.html\#str}{\sphinxstyleliteralemphasis{\sphinxupquote{str}}}) \textendash{} ruta de la capa

\end{description}\end{quote}

\end{fulllineitems}

\index{tipo\_clasificador\_s() (en el módulo apcsig)@\spxentry{tipo\_clasificador\_s()}\spxextra{en el módulo apcsig}}

\begin{fulllineitems}
\phantomsection\label{\detokenize{apcsig:apcsig.tipo_clasificador_s}}\pysiglinewithargsret{\sphinxcode{\sphinxupquote{apcsig.}}\sphinxbfcode{\sphinxupquote{tipo\_clasificador\_s}}}{\emph{clasificador}, \emph{path\_v}, \emph{l\_field}, \emph{campo\_cat=''}, \emph{fp=2}, \emph{categories=5}, \emph{min=0}, \emph{max=1}}{}~\begin{description}
\item[{Esta función integra los modos de clasificación, weber-fechner, progresiva,}] \leavevmode
cuartiles, quintiles, deciles o equidistante

param clasificador: tipo de clasificador (progresiva, cuartiles, quintiles, deciles, equidistante)
type clasificador: str
\begin{quote}\begin{description}
\item[{param path\_v}] \leavevmode
ruta de la capa vectorial

\item[{type path\_v}] \leavevmode
str

\item[{param l\_field}] \leavevmode
nombre del campo

\item[{type l\_field}] \leavevmode
str

\item[{param fp}] \leavevmode
factor de progresión

\item[{type fp}] \leavevmode
float

\item[{param categories}] \leavevmode
número de categorias

\item[{type categories}] \leavevmode
int

\item[{param min}] \leavevmode
valor mínimo de la capa

\item[{type min}] \leavevmode
float

\item[{param max}] \leavevmode
valor máximo de la capa

\item[{type max}] \leavevmode
float

\end{description}\end{quote}

\end{description}

\end{fulllineitems}

\index{vcopia() (en el módulo apcsig)@\spxentry{vcopia()}\spxextra{en el módulo apcsig}}

\begin{fulllineitems}
\phantomsection\label{\detokenize{apcsig:apcsig.vcopia}}\pysiglinewithargsret{\sphinxcode{\sphinxupquote{apcsig.}}\sphinxbfcode{\sphinxupquote{vcopia}}}{\emph{path\_vector}, \emph{path\_salida}}{}
Crea una copia de la capa a partir de la ruta de la capa,
la capa es creada con el mismo sistema de referencia que el origen.
\begin{quote}\begin{description}
\item[{Parámetros}] \leavevmode\begin{itemize}
\item {} 
\sphinxstyleliteralstrong{\sphinxupquote{path\_vector}} (\sphinxstyleliteralemphasis{\sphinxupquote{String}}) \textendash{} ruta de la capa original

\item {} 
\sphinxstyleliteralstrong{\sphinxupquote{path\_salida}} (\sphinxstyleliteralemphasis{\sphinxupquote{String}}) \textendash{} ruta de donde sera almacenada la capa

\end{itemize}

\end{description}\end{quote}

\end{fulllineitems}



\chapter{Análisis de sensibilidad}
\label{\detokenize{analisis:analisis-de-sensibilidad}}\label{\detokenize{analisis::doc}}
La prueba de sensibilidad por remoción de capas mide la importancia de cada mapa que se utiliza en un índice cartográfico como el que resulta de la aplicación de la combinación lineal ponderada.

Descargar el código de ejemplo

\sphinxcode{\sphinxupquote{sensibilidad.py}}.


\section{Requerimientos generales}
\label{\detokenize{analisis:requerimientos-generales}}
Para asegurar la ejecución correcta del código es importante
verificar la instalación y funcionamiento de los siguientes elementos:
\begin{itemize}
\item {} 
Qgis 3.4 o superior y librerías de Osgeo4W

\item {} 
Librerías python:

\end{itemize}
\begin{itemize}
\item {} 
copy

\item {} 
pprint

\item {} 
string

\item {} 
osgeo/gdal

\item {} 
gdal\_calc

\item {} 
os

\end{itemize}


\section{Requerimientos generales de los insumos}
\label{\detokenize{analisis:requerimientos-generales-de-los-insumos}}
Es importante que todas las capas raster cumplan con las siguientes condiciones:
\begin{itemize}
\item {} 
Misma proyección cartográfica

\item {} 
Mismo tamaño de pixel

\item {} 
Misma extensión de capa

\item {} 
Mismo valor de NoData

\end{itemize}


\section{Ejemplo}
\label{\detokenize{analisis:ejemplo}}

\subsection{Insumos}
\label{\detokenize{analisis:insumos}}
Crear una (1)carpeta en el directorio raíz o en la unidad C que se llame \sphinxstylestrong{analisis\_sensibilidad},
para descargar los insumos hacer clic \sphinxcode{\sphinxupquote{aqui}} (2)guarde el archivo \sphinxstylestrong{insumos.zip}
en la carpeta \sphinxstylestrong{analisis\_sensibilidad}, posteriormente hacer clic derecho sobre el archivo  y elegir la opción (3)Extract to insumos

\noindent\sphinxincludegraphics{{crear_carpeta}.JPG}

Una vez terminado el proceso, crear en la carpeta \sphinxstylestrong{analisis\_sensibilidad} una (4) carpeta con el nombre \sphinxstylestrong{salida}

\noindent\sphinxincludegraphics{{c_salida}.JPG}


\subsection{Procedimiento}
\label{\detokenize{analisis:procedimiento}}

\subsubsection{1. Abrir el código}
\label{\detokenize{analisis:abrir-el-codigo}}
Abrir el código \sphinxstylestrong{sensibilidad.py} en Qgis 3.4 o superior,
Para resolver cualquier duda al respecto, consultar la \sphinxhref{https://vichdzgeo.github.io/geo\_lancis/ejecucion.html}{guia}


\subsubsection{2. Actualizar el diccionario}
\label{\detokenize{analisis:actualizar-el-diccionario}}
Ingresar la (1) ponderación del compontente según corresponda (Exposición, Susceptibilidad, Resiliencia), posteriormente
ingresar la (2)ponderación del subcomponente (biológico,físico),
Ingresar el (3)nombre de la capa raster de entrada con su respectiva (4)ponderación y su (5)ruta
repita los pasos siguiendo la estructura y hasta ingresar cada una de las capas.

\noindent\sphinxincludegraphics{{diccionario}.JPG}


\subsubsection{3. Indicar el direcctorio de salida}
\label{\detokenize{analisis:indicar-el-direcctorio-de-salida}}
Indicar el directorio donde guardarán los archivos necesarios para realizar el análisis de sensibilidad
y el archivo  \sphinxstylestrong{analisis\_sensibilidad.csv} que contendrá los resultados.

\begin{sphinxVerbatim}[commandchars=\\\{\}]
\PYG{n}{p\PYGZus{}procesamiento} \PYG{o}{=} \PYG{l+s+s1}{\PYGZsq{}}\PYG{l+s+s1}{C:/analisis\PYGZus{}sensibilidad/salida/}\PYG{l+s+s1}{\PYGZsq{}}
\end{sphinxVerbatim}


\subsubsection{4. Ejecutar el código}
\label{\detokenize{analisis:ejecutar-el-codigo}}
hacer clic en el (1) botón de ejecutar código, puede demorar 10 minutos o más dependiendo el procesador y
memoria RAM que tenga el equipo en donde se ejecute, al concluir aparecerá en la (2) consola una lista que indica que
ha procesado cada una de las capas.

\noindent\sphinxincludegraphics{{ejecucion}.JPG}


\subsection{Bibliografía}
\label{\detokenize{analisis:bibliografia}}

\subsection{Documentación dentro del código}
\label{\detokenize{analisis:module-sensibilidad_por_remocion_capas}}\label{\detokenize{analisis:documentacion-dentro-del-codigo}}\index{sensibilidad\_por\_remocion\_capas (módulo)@\spxentry{sensibilidad\_por\_remocion\_capas}\spxextra{módulo}}
Autores: LANCIS -APC, Fidel Serrano,Victor Hernandez

Qgis 3.4 o superior
\index{crea\_capa() (en el módulo sensibilidad\_por\_remocion\_capas)@\spxentry{crea\_capa()}\spxextra{en el módulo sensibilidad\_por\_remocion\_capas}}

\begin{fulllineitems}
\phantomsection\label{\detokenize{analisis:sensibilidad_por_remocion_capas.crea_capa}}\pysiglinewithargsret{\sphinxcode{\sphinxupquote{sensibilidad\_por\_remocion\_capas.}}\sphinxbfcode{\sphinxupquote{crea\_capa}}}{\emph{ecuacion}, \emph{rasters\_input}, \emph{salida}}{}
Esta función crea una capa mediante la calculadora raster
de GDAL, esta función esta limitada hasta 14 variables en la ecuación.
\begin{quote}\begin{description}
\item[{Parámetros}] \leavevmode\begin{itemize}
\item {} 
\sphinxstyleliteralstrong{\sphinxupquote{ecuacion}} (\sphinxstyleliteralemphasis{\sphinxupquote{String}}) \textendash{} ecuación expresada en formato gdal,                    es este caso es la salida de la funcion \sphinxstyleemphasis{ecuacion\_clp}

\item {} 
\sphinxstyleliteralstrong{\sphinxupquote{rasters\_input}} (\sphinxstyleliteralemphasis{\sphinxupquote{lista}}) \textendash{} lista de los paths de los archivos rasters, salida de la función \sphinxstyleemphasis{separa\_ruta\_pesos}

\item {} 
\sphinxstyleliteralstrong{\sphinxupquote{salida}} (\sphinxstyleliteralemphasis{\sphinxupquote{String}}) \textendash{} ruta con extensión tiff de la salida

\end{itemize}

\end{description}\end{quote}

\end{fulllineitems}

\index{ecuacion\_clp() (en el módulo sensibilidad\_por\_remocion\_capas)@\spxentry{ecuacion\_clp()}\spxextra{en el módulo sensibilidad\_por\_remocion\_capas}}

\begin{fulllineitems}
\phantomsection\label{\detokenize{analisis:sensibilidad_por_remocion_capas.ecuacion_clp}}\pysiglinewithargsret{\sphinxcode{\sphinxupquote{sensibilidad\_por\_remocion\_capas.}}\sphinxbfcode{\sphinxupquote{ecuacion\_clp}}}{\emph{pesos}}{}
Esta función recibe una lista de pesos para regresar la ecuación
en la estructura requerida por gdal para la combinación lineal ponderada.
\begin{quote}\begin{description}
\item[{Parámetros}] \leavevmode
\sphinxstyleliteralstrong{\sphinxupquote{pesos}} (\sphinxstyleliteralemphasis{\sphinxupquote{lista}}) \textendash{} lista de los pesos de las capas, salida de la función \sphinxstyleemphasis{separa\_ruta\_pesos}

\end{description}\end{quote}

\end{fulllineitems}

\index{ecuacion\_vulnerabilidad() (en el módulo sensibilidad\_por\_remocion\_capas)@\spxentry{ecuacion\_vulnerabilidad()}\spxextra{en el módulo sensibilidad\_por\_remocion\_capas}}

\begin{fulllineitems}
\phantomsection\label{\detokenize{analisis:sensibilidad_por_remocion_capas.ecuacion_vulnerabilidad}}\pysiglinewithargsret{\sphinxcode{\sphinxupquote{sensibilidad\_por\_remocion\_capas.}}\sphinxbfcode{\sphinxupquote{ecuacion\_vulnerabilidad}}}{\emph{n}}{}
Esta función expresa la ecuación para el cálculo de la vulnerabilidad
\begin{align*}\!\begin{aligned}
vulnerabilidad = \exp^{( 1 - sus)^{(1 + ca)}}\\
| exp = Exposición
| sus = Susceptibilidad
| ca = Capacidad adaptativa\\
\end{aligned}\end{align*}\begin{quote}\begin{description}
\item[{Devuelve}] \leavevmode
str ecuacion

\end{description}\end{quote}

\end{fulllineitems}

\index{get\_region() (en el módulo sensibilidad\_por\_remocion\_capas)@\spxentry{get\_region()}\spxextra{en el módulo sensibilidad\_por\_remocion\_capas}}

\begin{fulllineitems}
\phantomsection\label{\detokenize{analisis:sensibilidad_por_remocion_capas.get_region}}\pysiglinewithargsret{\sphinxcode{\sphinxupquote{sensibilidad\_por\_remocion\_capas.}}\sphinxbfcode{\sphinxupquote{get\_region}}}{\emph{path\_layer}}{}
Esta función regresa en forma de cadena de texto
las coordenadas de la extensión de una capa raster

param path\_layer: ruta de la capa raster
type path\_layer: str

\end{fulllineitems}

\index{lista\_criterios() (en el módulo sensibilidad\_por\_remocion\_capas)@\spxentry{lista\_criterios()}\spxextra{en el módulo sensibilidad\_por\_remocion\_capas}}

\begin{fulllineitems}
\phantomsection\label{\detokenize{analisis:sensibilidad_por_remocion_capas.lista_criterios}}\pysiglinewithargsret{\sphinxcode{\sphinxupquote{sensibilidad\_por\_remocion\_capas.}}\sphinxbfcode{\sphinxupquote{lista\_criterios}}}{\emph{dicc}}{}
Esta función regresa una lista de los criterios de un diccionario
\begin{quote}\begin{description}
\item[{Parámetros}] \leavevmode
\sphinxstyleliteralstrong{\sphinxupquote{dicc}} \textendash{} Diccionario que contiene nombres, rutas y pesos para el

\end{description}\end{quote}

análisis de vulnerabilidad / sensibilidad
:type dicc: diccionario python

\end{fulllineitems}

\index{lista\_pesos\_ruta() (en el módulo sensibilidad\_por\_remocion\_capas)@\spxentry{lista\_pesos\_ruta()}\spxextra{en el módulo sensibilidad\_por\_remocion\_capas}}

\begin{fulllineitems}
\phantomsection\label{\detokenize{analisis:sensibilidad_por_remocion_capas.lista_pesos_ruta}}\pysiglinewithargsret{\sphinxcode{\sphinxupquote{sensibilidad\_por\_remocion\_capas.}}\sphinxbfcode{\sphinxupquote{lista\_pesos\_ruta}}}{\emph{dicc}}{}
Funcion para sacar listas por subcriterio

\end{fulllineitems}

\index{media\_raster() (en el módulo sensibilidad\_por\_remocion\_capas)@\spxentry{media\_raster()}\spxextra{en el módulo sensibilidad\_por\_remocion\_capas}}

\begin{fulllineitems}
\phantomsection\label{\detokenize{analisis:sensibilidad_por_remocion_capas.media_raster}}\pysiglinewithargsret{\sphinxcode{\sphinxupquote{sensibilidad\_por\_remocion\_capas.}}\sphinxbfcode{\sphinxupquote{media\_raster}}}{\emph{path\_raster}}{}
Esta función regresa el promedio de todos los pixeles válidos
de un archivo raster
\begin{quote}\begin{description}
\item[{Parámetros}] \leavevmode
\sphinxstyleliteralstrong{\sphinxupquote{path\_raster}} (\sphinxstyleliteralemphasis{\sphinxupquote{String}}) \textendash{} Ruta del archivo raster

\end{description}\end{quote}

\end{fulllineitems}

\index{nombre\_capa() (en el módulo sensibilidad\_por\_remocion\_capas)@\spxentry{nombre\_capa()}\spxextra{en el módulo sensibilidad\_por\_remocion\_capas}}

\begin{fulllineitems}
\phantomsection\label{\detokenize{analisis:sensibilidad_por_remocion_capas.nombre_capa}}\pysiglinewithargsret{\sphinxcode{\sphinxupquote{sensibilidad\_por\_remocion\_capas.}}\sphinxbfcode{\sphinxupquote{nombre\_capa}}}{\emph{path\_capa}}{}
Esta función regresa el nombre de una capa sin extensión
\begin{quote}\begin{description}
\item[{Parámetros}] \leavevmode
\sphinxstyleliteralstrong{\sphinxupquote{path\_capa}} (\sphinxhref{https://docs.python.org/3/library/stdtypes.html\#str}{\sphinxstyleliteralemphasis{\sphinxupquote{str}}}) \textendash{} ruta de la capa

\end{description}\end{quote}

\end{fulllineitems}

\index{quita() (en el módulo sensibilidad\_por\_remocion\_capas)@\spxentry{quita()}\spxextra{en el módulo sensibilidad\_por\_remocion\_capas}}

\begin{fulllineitems}
\phantomsection\label{\detokenize{analisis:sensibilidad_por_remocion_capas.quita}}\pysiglinewithargsret{\sphinxcode{\sphinxupquote{sensibilidad\_por\_remocion\_capas.}}\sphinxbfcode{\sphinxupquote{quita}}}{\emph{dicc}, \emph{key}}{}
Esta función retira un elemento del diccionario y regresa un nuevo diccionario
sin dicho elemento \textless{}\textless{}dicc\_q\textgreater{}\textgreater{}.
\begin{quote}\begin{description}
\item[{Parámetros}] \leavevmode\begin{itemize}
\item {} 
\sphinxstyleliteralstrong{\sphinxupquote{dicc}} (\sphinxstyleliteralemphasis{\sphinxupquote{diccionario}}) \textendash{} Diccionario con la estructura requerida

\item {} 
\sphinxstyleliteralstrong{\sphinxupquote{key}} (\sphinxstyleliteralemphasis{\sphinxupquote{String}}) \textendash{} nombre de la variable a quitar

\end{itemize}

\end{description}\end{quote}

\end{fulllineitems}

\index{quita\_reescala() (en el módulo sensibilidad\_por\_remocion\_capas)@\spxentry{quita\_reescala()}\spxextra{en el módulo sensibilidad\_por\_remocion\_capas}}

\begin{fulllineitems}
\phantomsection\label{\detokenize{analisis:sensibilidad_por_remocion_capas.quita_reescala}}\pysiglinewithargsret{\sphinxcode{\sphinxupquote{sensibilidad\_por\_remocion\_capas.}}\sphinxbfcode{\sphinxupquote{quita\_reescala}}}{\emph{dicc}, \emph{key}}{}
Función que integra las funciones quita y reescala y regresa
un diccionario sin la variable y con los pesos reescalados.
\begin{quote}\begin{description}
\item[{Parámetros}] \leavevmode\begin{itemize}
\item {} 
\sphinxstyleliteralstrong{\sphinxupquote{dicc}} (\sphinxstyleliteralemphasis{\sphinxupquote{diccionario}}) \textendash{} Diccionario con la estructura requerida

\item {} 
\sphinxstyleliteralstrong{\sphinxupquote{key}} (\sphinxstyleliteralemphasis{\sphinxupquote{String}}) \textendash{} nombre de la variable a quitar

\end{itemize}

\end{description}\end{quote}

\end{fulllineitems}

\index{raster\_min\_max() (en el módulo sensibilidad\_por\_remocion\_capas)@\spxentry{raster\_min\_max()}\spxextra{en el módulo sensibilidad\_por\_remocion\_capas}}

\begin{fulllineitems}
\phantomsection\label{\detokenize{analisis:sensibilidad_por_remocion_capas.raster_min_max}}\pysiglinewithargsret{\sphinxcode{\sphinxupquote{sensibilidad\_por\_remocion\_capas.}}\sphinxbfcode{\sphinxupquote{raster\_min\_max}}}{\emph{path\_raster}}{}
Esta funcion regresa los valores maximos y minimos de una capa raster
\begin{quote}\begin{description}
\item[{Parámetros}] \leavevmode
\sphinxstyleliteralstrong{\sphinxupquote{path\_raster}} (\sphinxhref{https://docs.python.org/3/library/stdtypes.html\#str}{\sphinxstyleliteralemphasis{\sphinxupquote{str}}}) \textendash{} ruta de la capa raster

\end{description}\end{quote}

\end{fulllineitems}

\index{reescala() (en el módulo sensibilidad\_por\_remocion\_capas)@\spxentry{reescala()}\spxextra{en el módulo sensibilidad\_por\_remocion\_capas}}

\begin{fulllineitems}
\phantomsection\label{\detokenize{analisis:sensibilidad_por_remocion_capas.reescala}}\pysiglinewithargsret{\sphinxcode{\sphinxupquote{sensibilidad\_por\_remocion\_capas.}}\sphinxbfcode{\sphinxupquote{reescala}}}{\emph{dicc\_q}}{}
Esta función rescala un diccionario que se le a quitado un criterio
y regresa el diccionario con los pesos rescalados
\begin{quote}\begin{description}
\item[{Parámetros}] \leavevmode
\sphinxstyleliteralstrong{\sphinxupquote{dicc\_q}} \textendash{} salida de la función \sphinxstyleemphasis{quita}

\end{description}\end{quote}

\end{fulllineitems}



\chapter{Índice Lee-Sallee}
\label{\detokenize{leesallee:indice-lee-sallee}}\label{\detokenize{leesallee::doc}}
Para capas vectoriales
\begin{equation*}
\begin{split}lee\_sallee\_index =  \frac{A\cap B}{A\cup B}\end{split}
\end{equation*}
descarga el código de ejemplo

\sphinxcode{\sphinxupquote{indice\_lee\_sallee.py}}.


\section{Ejemplo}
\label{\detokenize{leesallee:ejemplo}}

\subsection{Insumos}
\label{\detokenize{leesallee:insumos}}
Descarga los insumos para este ejemplo \sphinxcode{\sphinxupquote{aqui}}


\subsection{Procedimiento}
\label{\detokenize{leesallee:procedimiento}}
Abre el código \sphinxstylestrong{indice\_lee\_sallee.py} en Qgis 3.4 o superior,
Si tienes dudas de como hacerlo visualiza la \sphinxhref{https://vichdzgeo.github.io/geo\_lancis/ejecucion.html}{guia}

Modifica las rutas donde se encuentran los insumos
\begin{itemize}
\item {} 
vlayer\_base es la capa vectorial que se ocupa como base

\item {} 
vlayer\_model es la capa vectorial que se ocupa como modelo

\end{itemize}

\noindent\sphinxincludegraphics{{rutas}.png}
\phantomsection\label{\detokenize{leesallee:module-indice_lee_sallee}}\index{indice\_lee\_sallee (módulo)@\spxentry{indice\_lee\_sallee}\spxextra{módulo}}\index{indice\_lee\_salee() (en el módulo indice\_lee\_sallee)@\spxentry{indice\_lee\_salee()}\spxextra{en el módulo indice\_lee\_sallee}}

\begin{fulllineitems}
\phantomsection\label{\detokenize{leesallee:indice_lee_sallee.indice_lee_salee}}\pysiglinewithargsret{\sphinxcode{\sphinxupquote{indice\_lee\_sallee.}}\sphinxbfcode{\sphinxupquote{indice\_lee\_salee}}}{\emph{vlayer\_base}, \emph{vlayer\_model}, \emph{campo\_categoria}, \emph{categoria}, \emph{id='ageb\_id'}}{}~\begin{quote}

Esta función regresa el índice Lee-Sallee
\begin{equation*}
\begin{split}lee\_sallee\_index =  \end{split}
\end{equation*}\end{quote}

rac\{Acap B\}\{Acup B\}
\begin{quote}
\begin{quote}\begin{description}
\item[{param vlayer\_base}] \leavevmode
vector base

\item[{type vlayer\_base}] \leavevmode
QgsVectorLayer

\item[{param vlayer\_model}] \leavevmode
Vector modelo

\item[{type vlayer\_model}] \leavevmode
QgsVectorLayer

\item[{param campo\_categoria}] \leavevmode
Nombre del campo que tiene las categorias

\item[{type campo\_categoria}] \leavevmode
String

\item[{param categoria}] \leavevmode
Número de categoria

\item[{type categoria}] \leavevmode
int

\item[{param id}] \leavevmode
nombre del campo identificador

\item[{type id}] \leavevmode
String

\end{description}\end{quote}
\end{quote}

\end{fulllineitems}

\index{lista\_shp() (en el módulo indice\_lee\_sallee)@\spxentry{lista\_shp()}\spxextra{en el módulo indice\_lee\_sallee}}

\begin{fulllineitems}
\phantomsection\label{\detokenize{leesallee:indice_lee_sallee.lista_shp}}\pysiglinewithargsret{\sphinxcode{\sphinxupquote{indice\_lee\_sallee.}}\sphinxbfcode{\sphinxupquote{lista\_shp}}}{\emph{path\_carpeta}}{}~\begin{quote}\begin{description}
\item[{Parámetros}] \leavevmode
\sphinxstyleliteralstrong{\sphinxupquote{path\_carpeta}} (\sphinxstyleliteralemphasis{\sphinxupquote{String}}) \textendash{} ruta que contiene los archivos shape a procesar

\end{description}\end{quote}

\end{fulllineitems}



\chapter{OWA}
\label{\detokenize{owa:owa}}\label{\detokenize{owa::doc}}
OWA (Ordered Weighted Average) es un análisis de aptitud territorial basado en procedimientos de
Sistemas de Información Geográfica (SIG) y evaluación multicriterio (Malczewski, 2006).
El análisis OWA genera un amplio rango de escenarios de aptitud territorial cambiando únicamente un
parámetro lingüístico (alpha), relacionado con la rigidez en el cumplimiento de criterios preestablecidos.

OWA está definido por la siguiente ecuación:
\begin{equation*}
\begin{split}OWA=\sum_{j=1}^{n}\left (\left( \sum_{k=1}^{j}u_{k}\right )^{\alpha} - \left ( \sum_{k=1}^{j-1}u_{k}\right )^{\alpha} \right )z_{ij}\end{split}
\end{equation*}
Donde:

j = Criterio

uk = Peso ordenado del criterio j

k= Orden asignado al peso del criterio j (renglón)

i = Pixel

z\_ij = Valor ordenado del criterio j en el pixel i

\(\alpha\) = Cuantificador lingüístico

Descargar el código de ejemplo

\sphinxcode{\sphinxupquote{owa\_raster.py}}.


\section{Requerimientos generales}
\label{\detokenize{owa:requerimientos-generales}}
Para asegurar la ejecución correcta del código es importante
verificar la instalación y funcionamiento de los siguientes elementos:
\begin{itemize}
\item {} 
Qgis 3.4 o superior y librerías de Osgeo4W

\item {} 
Librerías python:

\end{itemize}
\begin{itemize}
\item {} 
Numpy

\item {} 
Pandas

\item {} 
GDAL

\item {} 
reduce

\end{itemize}


\section{Requerimientos generales de los insumos}
\label{\detokenize{owa:requerimientos-generales-de-los-insumos}}
Es importante que todas las capas raster cumplan con las siguientes condiciones:
\begin{itemize}
\item {} 
Misma proyección cartográfica

\item {} 
Mismo tamaño de pixel

\item {} 
Misma extensión de capa

\item {} 
Mismo valor de NoData

\end{itemize}


\section{Ejemplo}
\label{\detokenize{owa:ejemplo}}

\subsection{Insumos}
\label{\detokenize{owa:insumos}}
Descargar los insumos para este ejemplo \sphinxcode{\sphinxupquote{aqui}}


\subsection{Procedimiento}
\label{\detokenize{owa:procedimiento}}

\subsubsection{1. Abrir el código}
\label{\detokenize{owa:abrir-el-codigo}}
Abrir el código \sphinxstylestrong{owa\_raster.py} en Qgis 3.4 o superior,
Para resolver cualquier duda al respecto, consultar la \sphinxhref{https://vichdzgeo.github.io/geo\_lancis/ejecucion.html}{guia}

\noindent\sphinxincludegraphics{{codigo1}.PNG}


\subsubsection{2. Actualizar el diccionario}
\label{\detokenize{owa:actualizar-el-diccionario}}
Ingresar las capas raster de entrada con sus respectivos pesos
a la función mediante un diccionario. Es importante seguir la
estructura del siguiente ejemplo:

\begin{sphinxVerbatim}[commandchars=\\\{\}]
\PYG{n}{dicc\PYGZus{}capas} \PYG{o}{=} \PYG{p}{\PYGZob{}}\PYG{l+s+s1}{\PYGZsq{}}\PYG{l+s+s1}{capa\PYGZus{}1}\PYG{l+s+s1}{\PYGZsq{}}\PYG{p}{:}\PYG{p}{\PYGZob{}}\PYG{l+s+s1}{\PYGZsq{}}\PYG{l+s+s1}{ruta}\PYG{l+s+s1}{\PYGZsq{}}\PYG{p}{:}\PYG{l+s+s2}{\PYGZdq{}}\PYG{l+s+s2}{C:/Dropbox (LANCIS)/SIG/desarrollo/sig\PYGZus{}papiit/entregables/exposicion/biologica/v\PYGZus{}acuatica\PYGZus{}yuc/fv\PYGZus{}v\PYGZus{}acuatica\PYGZus{}yuc.tif}\PYG{l+s+s2}{\PYGZdq{}}\PYG{p}{,}\PYG{l+s+s1}{\PYGZsq{}}\PYG{l+s+s1}{w}\PYG{l+s+s1}{\PYGZsq{}}\PYG{p}{:}\PYG{l+m+mf}{0.08}\PYG{p}{\PYGZcb{}}\PYG{p}{,}
        \PYG{l+s+s1}{\PYGZsq{}}\PYG{l+s+s1}{capa\PYGZus{}2}\PYG{l+s+s1}{\PYGZsq{}}\PYG{p}{:}\PYG{p}{\PYGZob{}}\PYG{l+s+s1}{\PYGZsq{}}\PYG{l+s+s1}{ruta}\PYG{l+s+s1}{\PYGZsq{}}\PYG{p}{:}\PYG{l+s+s2}{\PYGZdq{}}\PYG{l+s+s2}{C:/Dropbox (LANCIS)/SIG/desarrollo/sig\PYGZus{}papiit/entregables/exposicion/biologica/v\PYGZus{}costera\PYGZus{}yuc/fv\PYGZus{}v\PYGZus{}costera\PYGZus{}distancia\PYGZus{}yuc.tif}\PYG{l+s+s2}{\PYGZdq{}}\PYG{p}{,}\PYG{l+s+s1}{\PYGZsq{}}\PYG{l+s+s1}{w}\PYG{l+s+s1}{\PYGZsq{}}\PYG{p}{:}\PYG{l+m+mf}{0.42}\PYG{p}{\PYGZcb{}}\PYG{p}{,}
        \PYG{l+s+s1}{\PYGZsq{}}\PYG{l+s+s1}{capa\PYGZus{}3}\PYG{l+s+s1}{\PYGZsq{}}\PYG{p}{:}\PYG{p}{\PYGZob{}}\PYG{l+s+s1}{\PYGZsq{}}\PYG{l+s+s1}{ruta}\PYG{l+s+s1}{\PYGZsq{}}\PYG{p}{:}\PYG{l+s+s2}{\PYGZdq{}}\PYG{l+s+s2}{C:/Dropbox (LANCIS)/SIG/desarrollo/sig\PYGZus{}papiit/entregables/exposicion/fisica/ancho\PYGZus{}playa\PYGZus{}yuc/fv\PYGZus{}distancia\PYGZus{}playa\PYGZus{}yuc.tif}\PYG{l+s+s2}{\PYGZdq{}}\PYG{p}{,}\PYG{l+s+s1}{\PYGZsq{}}\PYG{l+s+s1}{w}\PYG{l+s+s1}{\PYGZsq{}}\PYG{p}{:}\PYG{l+m+mf}{0.065}\PYG{p}{\PYGZcb{}}\PYG{p}{,}
        \PYG{l+s+s1}{\PYGZsq{}}\PYG{l+s+s1}{capa\PYGZus{}4}\PYG{l+s+s1}{\PYGZsq{}}\PYG{p}{:}\PYG{p}{\PYGZob{}}\PYG{l+s+s1}{\PYGZsq{}}\PYG{l+s+s1}{ruta}\PYG{l+s+s1}{\PYGZsq{}}\PYG{p}{:}\PYG{l+s+s2}{\PYGZdq{}}\PYG{l+s+s2}{C:/Dropbox (LANCIS)/SIG/desarrollo/sig\PYGZus{}papiit/entregables/exposicion/fisica/elev\PYGZus{}yuc/fv\PYGZus{}elevacion\PYGZus{}yuc.tif}\PYG{l+s+s2}{\PYGZdq{}}\PYG{p}{,}\PYG{l+s+s1}{\PYGZsq{}}\PYG{l+s+s1}{w}\PYG{l+s+s1}{\PYGZsq{}}\PYG{p}{:}\PYG{l+m+mf}{0.435}\PYG{p}{\PYGZcb{}}\PYG{p}{,}
        \PYG{p}{\PYGZcb{}}
\end{sphinxVerbatim}

Donde:
\begin{itemize}
\item {} 
\sphinxstylestrong{capa\_\#}:  Corresponde a la capa en el orden en que se agregó al diccionario,

\item {} 
\sphinxstylestrong{ruta} : Corresponde a la ruta o path de la capa

\item {} 
\sphinxstylestrong{w} : Corresponde al peso asociado a esa capa o criterio

\end{itemize}

\begin{sphinxadmonition}{note}{Nota:}
Para adicionar una capa, agregar el consecutivo
a la llave de la capa (en este caso capa\_5).
La línea quedaría de la siguiente forma:

“capa\_5”:\{“ruta”:path\_tiff,”w”:\#.\#\#\#\},
\}
\end{sphinxadmonition}


\subsubsection{3. Indicar la capa maestra}
\label{\detokenize{owa:indicar-la-capa-maestra}}
Para generar la salida en formato tiff se requiere conocer aspectos
técnicos como número de columnas y renglones, tamaño de pixel, coordenadas
del extent, entre otros.

Estos datos son extraidos por el código mediante la variable \sphinxstylestrong{path\_capa\_maestra},
en ella, se indica la ruta de \sphinxstylestrong{cualquier} capa raster ingresada en el diccionario del
paso \#2.

como ejemplo se toma la ruta de la \sphinxstylestrong{capa\_1}

\begin{sphinxVerbatim}[commandchars=\\\{\}]
\PYG{n}{path\PYGZus{}capa\PYGZus{}maestra} \PYG{o}{=} \PYG{l+s+s2}{\PYGZdq{}}\PYG{l+s+s2}{C:/Dropbox (LANCIS)/SIG/desarrollo/sig\PYGZus{}papiit/entregables/exposicion/biologica/v\PYGZus{}acuatica\PYGZus{}yuc/fv\PYGZus{}v\PYGZus{}acuatica\PYGZus{}yuc.tif}\PYG{l+s+s2}{\PYGZdq{}}
\end{sphinxVerbatim}


\subsubsection{4. Indicar el direcctorio de salida}
\label{\detokenize{owa:indicar-el-direcctorio-de-salida}}
Indicar el directorio donde guardarán los mapas de salida.

por ejemplo:

\begin{sphinxVerbatim}[commandchars=\\\{\}]
\PYG{n}{path\PYGZus{}salida} \PYG{o}{=} \PYG{l+s+s2}{\PYGZdq{}}\PYG{l+s+s2}{C:/Dropbox (LANCIS)/SIG/desarrollo/sig\PYGZus{}papiit/procesamiento/owa/}\PYG{l+s+s2}{\PYGZdq{}}
\end{sphinxVerbatim}


\subsubsection{5. Los valores de alpha}
\label{\detokenize{owa:los-valores-de-alpha}}
El código tiene valores  predeterminados de alpha

\begin{sphinxadmonition}{note}{Nota:}
Para más información respecto a los valores de alpha consulte la
bibliografía
\end{sphinxadmonition}

\begin{sphinxVerbatim}[commandchars=\\\{\}]
\PYG{n}{owa\PYGZus{}alphas} \PYG{o}{=} \PYG{p}{[}\PYG{l+m+mf}{0.0001}\PYG{p}{,}\PYG{l+m+mf}{0.1}\PYG{p}{,}\PYG{l+m+mf}{0.5}\PYG{p}{,}\PYG{l+m+mf}{1.0}\PYG{p}{,}\PYG{l+m+mf}{2.0}\PYG{p}{,}\PYG{l+m+mf}{10.0}\PYG{p}{,}\PYG{l+m+mf}{1000.0}\PYG{p}{]}
\end{sphinxVerbatim}


\begin{savenotes}\sphinxattablestart
\centering
\begin{tabulary}{\linewidth}[t]{|T|T|}
\hline
\sphinxstyletheadfamily 
\(\alpha\)
&\sphinxstyletheadfamily 
Quantifier (Q)
\\
\hline
0.0001
&
At least one
\\
\hline
0.1
&
At least a few a a
\\
\hline
0.5
&
A few
\\
\hline
1.0
&
Half (identity)
\\
\hline
2.0
&
Most
\\
\hline
10.0
&
Almost all
\\
\hline
1000
&
All
\\
\hline
\end{tabulary}
\par
\sphinxattableend\end{savenotes}

para cada valor en la lista, el código generará un mapa en el directorio
de salida

\noindent\sphinxincludegraphics{{salida}.png}


\section{Bibliografía}
\label{\detokenize{owa:bibliografia}}
Malczewski, J. (2006). Ordered weighted averaging with fuzzy quantifiers:
GIS-based multicriteria evaluation for land-use suitability analysis.
International Journal of Applied Earth Observation and Geoin-formation, 8 ,270-277.


\section{Documentación dentro del código}
\label{\detokenize{owa:documentacion-dentro-del-codigo}}

\chapter{Verificación de capas}
\label{\detokenize{verificacion:verificacion-de-capas}}\label{\detokenize{verificacion::doc}}
Los puntos a verificar son los siguientes:
\begin{enumerate}
\def\theenumi{\arabic{enumi}}
\def\labelenumi{\theenumi .}
\makeatletter\def\p@enumii{\p@enumi \theenumi .}\makeatother
\item {} 
Proyección Que exista el archivo prj asociado

\item {} 
Geometría completa  Que exista los mismos elementos geométricos que los contenidos en la tabla de atributos

\item {} 
Sobrelapados Que la capa no cuente con errores topológicos

\item {} 
Nulos Que no existan campos vacios en la tabla de atributos

\item {} 
Codificados Que no existan campos con caracteres espeaciales o extraños en su contenido o que datos numéricos esten declardos como texto

\item {} 
Metadatos Que exista el archivo xml asociado a los metadatos geográficos

\end{enumerate}

Descargar el código de ejemplo

\sphinxcode{\sphinxupquote{verificacion\_layers.py}}.


\section{Requerimientos generales}
\label{\detokenize{verificacion:requerimientos-generales}}

\section{Ejemplo:}
\label{\detokenize{verificacion:ejemplo}}

\subsection{1. Abrir el código}
\label{\detokenize{verificacion:abrir-el-codigo}}
Abrir el código \sphinxstylestrong{verificación\_layers.py} en Qgis 3.4 o superior,
para resolver cualquier duda al respecto, consultar la \sphinxhref{https://vichdzgeo.github.io/geo\_lancis/ejecucion.html}{guia}

\noindent\sphinxincludegraphics{{codigo3}.PNG}


\subsection{2. Indicar el directorio}
\label{\detokenize{verificacion:indicar-el-directorio}}
El código verifica todas las capas vectoriales contenidas en el directorio
indicado en la variable \sphinxstylestrong{path\_dir}

\begin{sphinxVerbatim}[commandchars=\\\{\}]
\PYG{n}{path\PYGZus{}dir} \PYG{o}{=} \PYG{l+s+s2}{\PYGZdq{}}\PYG{l+s+s2}{C:/Dropbox (LANCIS)/SIG/insumos/agricultura/conabio/vector/produccion\PYGZus{}miel/}\PYG{l+s+s2}{\PYGZdq{}}
\end{sphinxVerbatim}


\subsection{Salidas}
\label{\detokenize{verificacion:salidas}}

\subsubsection{Capas de topología}
\label{\detokenize{verificacion:capas-de-topologia}}
el código crea una carpeta llamada \sphinxstylestrong{temp}, dentro de ella otra carpeta llamada \sphinxstylestrong{topologia} en esta
carpeta se guardan los 3 archivos shapefile  resultantes de la función \sphinxstylestrong{topologia}

estan nombrados de la siguiente manera:

\sphinxstylestrong{nombrecapa\_error}.shp

Capa de puntos que índica la posición en donde 2 o más poligonos no colindan adecuadamente
o que su geometría puede causar problemas en algún análisis espacial

\sphinxstylestrong{nombrecapa\_invalido}.shp

Capa de poligonos que índica geométria inválida de la capa

\sphinxstylestrong{nombrecapa\_valido}.shp

Capa de poligonos que indica la geometría válida de la capa


\subsubsection{Imagen de la capa}
\label{\detokenize{verificacion:imagen-de-la-capa}}
\begin{sphinxadmonition}{note}{Nota:}
El código genera una imagen de la capa con el mapa base de openstreetmap
sí y solo sí la capa tiene un puntaje de 10 en el criterio de \sphinxstylestrong{proyección}
\end{sphinxadmonition}


\chapter{Tabular intersección entre 3 geometrías}
\label{\detokenize{tabulacion_3geo:tabular-interseccion-entre-3-geometrias}}\label{\detokenize{tabulacion_3geo::doc}}
Calcula la intersección de 3 capas y realiza una tabulación cruzada del área

La capa A es al nivel geográfico que se reporta
La capa B es el nivel geométrico intermedio del cual se cuantifica
el área total perteneciente a la entidad A y se cuantifica el área  por clase de USV.
La capa C es la serie de uso de suelo y vegetación de INEGI


\section{Requerimientos generales}
\label{\detokenize{tabulacion_3geo:requerimientos-generales}}
Para asegurar la ejecución correcta del código es importante
verificar la instalación y funcionamiento de los siguientes elementos:
\begin{itemize}
\item {} 
Qgis 3.4 o superior

\end{itemize}

Descargar el código de ejemplo

\sphinxcode{\sphinxupquote{tabulacion\_3geo.py}}.


\section{Requerimientos generales de los insumos}
\label{\detokenize{tabulacion_3geo:requerimientos-generales-de-los-insumos}}
Es importante que todas las capas vectoriales cumplan con las siguientes condiciones:
\begin{itemize}
\item {} 
Misma proyección cartográfica

\item {} 
Sin problemas topológicos

\end{itemize}


\subsection{Entidad A}
\label{\detokenize{tabulacion_3geo:entidad-a}}\begin{itemize}
\item {} 
Que cuente con un campo que contenga un identificador único para cada geométria

\end{itemize}


\subsection{Entidad B}
\label{\detokenize{tabulacion_3geo:entidad-b}}\begin{itemize}
\item {} 
Que cuente con un campo dónde se especifiquen las categorias o clases

\end{itemize}


\subsection{Entidad C}
\label{\detokenize{tabulacion_3geo:entidad-c}}\begin{itemize}
\item {} 
Que cuente con un campo que contenga el número de clase correspondiente

\item {} 
Que el campo mencionado en el inciso anterior se llame igual para todas las series

\end{itemize}


\section{Ejemplo}
\label{\detokenize{tabulacion_3geo:ejemplo}}

\subsection{Datos de prueba}
\label{\detokenize{tabulacion_3geo:datos-de-prueba}}
Descargar los datos de prueba para este ejemplo \sphinxcode{\sphinxupquote{aqui}}


\begin{savenotes}\sphinxattablestart
\centering
\begin{tabulary}{\linewidth}[t]{|T|T|}
\hline
\sphinxstyletheadfamily 
insumo
&\sphinxstyletheadfamily 
Descripción
\\
\hline
muni\_2018\_utm16.shp
&
Capa de municipios del estado de Yucatán, esta capa representa la entidad A
\\
\hline
degradacion\_suelo\_yuc.shp
&
Capa de áreas de degradación del suelo clasificadas en ligero, moderado, alto y extremo, esta capa representa la entidad B
\\
\hline
usv/
&
Directorio que contiene capas de USV de las 6 series de INEGI para el estado de Yucatán, estas capas representan la entidad C
\\
\hline
\end{tabulary}
\par
\sphinxattableend\end{savenotes}


\subsubsection{Abrir el código}
\label{\detokenize{tabulacion_3geo:abrir-el-codigo}}
Abrir el código \sphinxstylestrong{tabulacion\_3geo.py} en Qgis 3.4 o superior,
Para resolver cualquier duda al respecto, consultar la \sphinxhref{https://vichdzgeo.github.io/geo\_lancis/ejecucion.html}{guia}

\noindent\sphinxincludegraphics{{codigo2}.PNG}


\subsubsection{1. Indicar la ruta de los insumos}
\label{\detokenize{tabulacion_3geo:indicar-la-ruta-de-los-insumos}}
Indicar la ruta completa de los insumos según corresponda:

\begin{sphinxadmonition}{warning}{Advertencia:}
verifica que se use “/” como separadador de espacio en lugar de “'
\end{sphinxadmonition}

\begin{sphinxVerbatim}[commandchars=\\\{\}]
\PYG{n}{path\PYGZus{}geometria\PYGZus{}a} \PYG{o}{=} \PYG{l+s+s1}{\PYGZsq{}}\PYG{l+s+s1}{C:/geo\PYGZus{}lancis/tabulacion3geo/insumos/muni\PYGZus{}2018\PYGZus{}utm16.shp}\PYG{l+s+s1}{\PYGZsq{}}
\PYG{n}{path\PYGZus{}geometria\PYGZus{}b} \PYG{o}{=} \PYG{l+s+s2}{\PYGZdq{}}\PYG{l+s+s2}{C:/geo\PYGZus{}lancis/tabulacion3geo/insumos/degradacion\PYGZus{}suelo\PYGZus{}yuc.shp}\PYG{l+s+s2}{\PYGZdq{}}
\PYG{n}{path\PYGZus{}dir\PYGZus{}usv} \PYG{o}{=} \PYG{l+s+s2}{\PYGZdq{}}\PYG{l+s+s2}{C:/geo\PYGZus{}lancis/tabulacion3geo/insumos/usv/}\PYG{l+s+s2}{\PYGZdq{}}
\end{sphinxVerbatim}


\subsubsection{2. Indicar los nombres de los campos id}
\label{\detokenize{tabulacion_3geo:indicar-los-nombres-de-los-campos-id}}
Para la geometría A se declara el nombre del campo identificador o clave

\begin{sphinxVerbatim}[commandchars=\\\{\}]
\PYG{n}{campo\PYGZus{}id\PYGZus{}geometria\PYGZus{}a} \PYG{o}{=} \PYG{l+s+s1}{\PYGZsq{}}\PYG{l+s+s1}{cve\PYGZus{}mun}\PYG{l+s+s1}{\PYGZsq{}}
\end{sphinxVerbatim}

\begin{sphinxadmonition}{note}{Nota:}
El nombre del campo para la geometría B se preguntará más adelante
por medio de una ventana emergente. ver paso 6
\end{sphinxadmonition}

Para la geometría C se declara el nombre del campo el cúal contiene el identificador
de clase para las series de USV

\begin{sphinxVerbatim}[commandchars=\\\{\}]
\PYG{n}{campo\PYGZus{}clase\PYGZus{}usv} \PYG{o}{=} \PYG{l+s+s1}{\PYGZsq{}}\PYG{l+s+s1}{id\PYGZus{}clase2}\PYG{l+s+s1}{\PYGZsq{}}
\end{sphinxVerbatim}


\subsubsection{4. Agrega identificadores según el tipo de geometría}
\label{\detokenize{tabulacion_3geo:agrega-identificadores-segun-el-tipo-de-geometria}}
En la variable \sphinxstylestrong{clave\_capas} agragar un identificador  para cada una de las
categorias de tres caracteres.

\begin{sphinxVerbatim}[commandchars=\\\{\}]
\PYG{n}{clave\PYGZus{}capas}\PYG{o}{=}\PYG{p}{[}\PYG{l+s+s1}{\PYGZsq{}}\PYG{l+s+s1}{mun}\PYG{l+s+s1}{\PYGZsq{}}\PYG{p}{,}\PYG{l+s+s1}{\PYGZsq{}}\PYG{l+s+s1}{deg}\PYG{l+s+s1}{\PYGZsq{}}\PYG{p}{,}\PYG{l+s+s1}{\PYGZsq{}}\PYG{l+s+s1}{usv}\PYG{l+s+s1}{\PYGZsq{}}\PYG{p}{]}
\end{sphinxVerbatim}


\subsubsection{4. Indica el directorio de salida}
\label{\detokenize{tabulacion_3geo:indica-el-directorio-de-salida}}
En esta ruta se escribirá la tabla resultado de la cruza adicionalmente se  crea una carpeta llamada \sphinxstylestrong{tmp} en la cúal se almacenán
las cruzas ralizadas en el proceso.

\begin{sphinxVerbatim}[commandchars=\\\{\}]
\PYG{n}{path\PYGZus{}salida} \PYG{o}{=} \PYG{l+s+s2}{\PYGZdq{}}\PYG{l+s+s2}{C:/geo\PYGZus{}lancis/tabulacion3geo/}\PYG{l+s+s2}{\PYGZdq{}}
\end{sphinxVerbatim}


\subsubsection{5. Ejecuta el script}
\label{\detokenize{tabulacion_3geo:ejecuta-el-script}}
Hacer clic en el botón de ejecutar y permanece atento a la consola.


\subsubsection{6. Ingresa el nombre del campo identificador de la geometría B}
\label{\detokenize{tabulacion_3geo:ingresa-el-nombre-del-campo-identificador-de-la-geometria-b}}
Se mostrará en la consola la siguiente instrucción:

\sphinxstyleemphasis{Ingrese el nombre del campo de las categorias a elegir en la geometría B}

enseguida del nombre de todos los campos que tiene la capa, escriba en la
ventana de texto el nombre del campo tal cual se muestra en la consola,
para este ejemplo el campo que sirve como identificador es \sphinxstylestrong{GRADO}  una vez escrito
hacer clic en \sphinxstylestrong{OK}

\noindent\sphinxincludegraphics{{campob}.PNG}


\subsubsection{7. Ingresar el número de las categorías a considerar}
\label{\detokenize{tabulacion_3geo:ingresar-el-numero-de-las-categorias-a-considerar}}
El script permite ingresar todas las categorías o solo algunas, las categorias
indicadas serán reclasificadas como binario donde 1 es que fue considerada y 0 cero que no,
posteriormente realiza una limpieza de la capa eliminando las geométrias con la categoria 0,
una vez finalizado ese proceso, realiza la cruza solo con las categorias consideredas.

Ingrese las categorias que desea considerar conforme se muestra en la consola, para
este ejemplo solo consideramos las categorias “Moderado”, “Fuerte” y “Extremo”, por lo
cual en la ventana que se muestra se escribe: 1,2,4, una vez escrito dar clic en \sphinxstylestrong{OK}

\begin{sphinxadmonition}{note}{Nota:}
puede ingresar solo una categoria escribendo por ejemplo: 1,

Puede ingresar todas las categorias escribiendo la palabra «TODAS»
\end{sphinxadmonition}

\noindent\sphinxincludegraphics{{categorias}.PNG}


\subsubsection{8.- Elegir entre área o porcentaje}
\label{\detokenize{tabulacion_3geo:elegir-entre-area-o-porcentaje}}
La cuantificación de las categorias de USV en el área de la geometría B perteneciente
a una identidad de la geometría A, puede expresarse en área (hectáreas) o en porcentaje.
para este ejemplo se elige “area”

\noindent\sphinxincludegraphics{{area}.PNG}


\chapter{Cobertura por niveles areas}
\label{\detokenize{nivel_geometrias:cobertura-por-niveles-areas}}\label{\detokenize{nivel_geometrias::doc}}
El objetivo de esta herramienta es cuantificar el área según el tipo de clase
de uso de suelo vegetación asociado a una geometría intermedia.

para comprender mejor lo anterior se expresa el siguiente caso

se quiere conocer a nivel de municipio (geometría A), el espacio designado como \sphinxstylestrong{Área Natural Protegida}
(geometria B), en dicho espacio, se requiere cuantificar el tipo de cobertura por clase
de la serie de uso de suelo y vegetación de INEGI (Geometría C).

Por lo tanto, se tendrá como resultado una capa geografica a nivel municipio que en su tabla de atributos
cuente con:
\begin{itemize}
\item {} 
Todos los campos de la capa original de municipios

\item {} 
un campo del área total (expresado en hectáreas) del espacio designado como \sphinxstylestrong{Área Natural Protegida} perteneciente al municipio,

\item {} 
campos nombrados como \sphinxstylestrong{clase\_\#} donde \sphinxstylestrong{\#} corresponde al número de clase o cobertura asociado a la capa de uso de suelo y vegetación. Estos campos pueden contener el área por clase expresada en hectáreas, o bien, el porcentaje correspondiente al área total del espacio designado como \sphinxstylestrong{Área Natural Protegida} perteneciente al municipio.

\end{itemize}


\section{Requerimientos generales}
\label{\detokenize{nivel_geometrias:requerimientos-generales}}
Para asegurar la ejecución correcta del código es importante
verificar la instalación y funcionamiento de los siguientes elementos:
\begin{itemize}
\item {} 
Qgis 3.4 o superior

\end{itemize}


\section{Requerimientos generales de los insumos}
\label{\detokenize{nivel_geometrias:requerimientos-generales-de-los-insumos}}
Es importante que todas las capas vectoriales cumplan con las siguientes condiciones:
\begin{itemize}
\item {} 
Misma proyección UTM

\item {} 
Sin problemas topólogicos

\end{itemize}


\subsection{Geometría A}
\label{\detokenize{nivel_geometrias:geometria-a}}\begin{itemize}
\item {} 
Tener un campo que contenga un identificador único para cada geometría (puede ser de tipo texto o entero)

\end{itemize}


\subsection{Geometría B}
\label{\detokenize{nivel_geometrias:geometria-b}}\begin{itemize}
\item {} 
Tener un campo que contenga las diferentes categorias o tipos de geometria (puede ser de tipo texto o entero)

\end{itemize}


\subsection{Geometría C}
\label{\detokenize{nivel_geometrias:geometria-c}}\begin{itemize}
\item {} 
Tener el siguiente tipo de nombrado
\begin{quote}

\sphinxstylestrong{usv\_serie\#\_aaa.shp}

Donde:
\begin{itemize}
\item {} 
\sphinxstylestrong{\#} representa el número de serie (1,2,3,4,5 o 6) \sphinxstylestrong{obligatorio}

\item {} 
\sphinxstylestrong{\_aaa} representa una abreviatura del lugar, para este ejemplo se ocupa  \sphinxstylestrong{\_yuc}

\end{itemize}
\end{quote}

\item {} 
Tener un campo de tipo entero que contenga las diferentes clases de cobertura empezando por 1, este campo debe estar presente en todas las capas de USV y debe llamarse de la misma forma

\end{itemize}


\section{Ejemplo}
\label{\detokenize{nivel_geometrias:ejemplo}}

\subsection{Insumos}
\label{\detokenize{nivel_geometrias:insumos}}
Descargar los insumos para este ejemplo \sphinxcode{\sphinxupquote{aqui}}


\subsubsection{1. Abrir el código}
\label{\detokenize{nivel_geometrias:abrir-el-codigo}}
Abrir el código \sphinxstylestrong{owa\_raster.py} en Qgis 3.4 o superior,
Para resolver cualquier duda al respecto, consultar la \sphinxhref{https://vichdzgeo.github.io/geo\_lancis/ejecucion.html}{guia}

\noindent\sphinxincludegraphics{{codigo}.PNG}


\subsubsection{2. Ingresar la ruta de la geometria A}
\label{\detokenize{nivel_geometrias:ingresar-la-ruta-de-la-geometria-a}}
Se ingresa la ruta completa de la capa de \sphinxstylestrong{municipios} en la variable
\sphinxstylestrong{path\_geometria\_a}

\begin{sphinxVerbatim}[commandchars=\\\{\}]
\PYG{n}{path\PYGZus{}geometria\PYGZus{}a} \PYG{o}{=} \PYG{l+s+s1}{\PYGZsq{}}\PYG{l+s+s1}{C:/Dropbox (LANCIS)/SIG/desarrollo/sig\PYGZus{}fomix/entregables/mun\PYGZus{}region/muni\PYGZus{}2018\PYGZus{}utm16.shp}\PYG{l+s+s1}{\PYGZsq{}}
\end{sphinxVerbatim}


\subsubsection{3. Ingresar la ruta de la geometria B}
\label{\detokenize{nivel_geometrias:ingresar-la-ruta-de-la-geometria-b}}
Se ingresa la ruta completa de la capa de \sphinxstylestrong{Áreas Naturales Protegidas} en la variable
\sphinxstylestrong{path\_geometria\_b}

\begin{sphinxVerbatim}[commandchars=\\\{\}]
\PYG{n}{path\PYGZus{}geometria\PYGZus{}b} \PYG{o}{=} \PYG{l+s+s2}{\PYGZdq{}}\PYG{l+s+s2}{C:/Dropbox (LANCIS)/SIG/desarrollo/sig\PYGZus{}fomix/insumos/ambiente/sds/anps\PYGZus{}sds/anps.shp}\PYG{l+s+s2}{\PYGZdq{}}
\end{sphinxVerbatim}


\subsubsection{4. Ingresar el directorio de las series de USV (geometría C)}
\label{\detokenize{nivel_geometrias:ingresar-el-directorio-de-las-series-de-usv-geometria-c}}
Se ingresa la ruta del directorio donde se encutran las capas
de \sphinxstylestrong{Uso de suelo y Vegetación ** en la variable
**path\_dir\_usv}

\begin{sphinxVerbatim}[commandchars=\\\{\}]
\PYG{n}{path\PYGZus{}dir\PYGZus{}usv} \PYG{o}{=} \PYG{l+s+s2}{\PYGZdq{}}\PYG{l+s+s2}{C:/Dropbox (LANCIS)/SIG/desarrollo/sig\PYGZus{}fomix/entregables/usv\PYGZus{}v2/}\PYG{l+s+s2}{\PYGZdq{}}
\end{sphinxVerbatim}


\subsubsection{5. Ingresar el directorio de salida}
\label{\detokenize{nivel_geometrias:ingresar-el-directorio-de-salida}}
Se ingresa la ruta del directorio de salida de los datos en la variable
** path\_salida**

\begin{sphinxVerbatim}[commandchars=\\\{\}]
\PYG{n}{path\PYGZus{}salida} \PYG{o}{=} \PYG{l+s+s2}{\PYGZdq{}}\PYG{l+s+s2}{C:/Dropbox (LANCIS)/SIG/desarrollo/sig\PYGZus{}fomix/procesamiento/municipios\PYGZus{}anp/}\PYG{l+s+s2}{\PYGZdq{}}
\end{sphinxVerbatim}

en esta carpeta estarán los resultados del script, tambien se conservan los datos
intermedios o productos de las intersecciones realizadas para la consulta de las áreas
en una subcarpeta llamada \sphinxstylestrong{tmp}


\subsubsection{6. Ingresar el nombre del campo ID de la capa A}
\label{\detokenize{nivel_geometrias:ingresar-el-nombre-del-campo-id-de-la-capa-a}}
se Ingresa el nombre del campo que contiene el identificador único
para las geometrías en la variable \sphinxstylestrong{campo\_id\_geometria\_a}, en este caso el nombre del campo  es \sphinxstylestrong{cve\_mun}

\begin{sphinxVerbatim}[commandchars=\\\{\}]
\PYG{n}{campo\PYGZus{}id\PYGZus{}geomatria\PYGZus{}a} \PYG{o}{=} \PYG{l+s+s1}{\PYGZsq{}}\PYG{l+s+s1}{cve\PYGZus{}mun}\PYG{l+s+s1}{\PYGZsq{}}
\end{sphinxVerbatim}


\subsubsection{7. Ingresar el nombre del campo de las categorias USV}
\label{\detokenize{nivel_geometrias:ingresar-el-nombre-del-campo-de-las-categorias-usv}}
Se ingresa el nombre del campo que contiene las categorias de USV en la
variable ** campo\_clase\_usv**

\begin{sphinxVerbatim}[commandchars=\\\{\}]
\PYG{n}{campo\PYGZus{}clase\PYGZus{}usv} \PYG{o}{=} \PYG{l+s+s1}{\PYGZsq{}}\PYG{l+s+s1}{id\PYGZus{}clase2}\PYG{l+s+s1}{\PYGZsq{}}
\end{sphinxVerbatim}


\subsubsection{8. Ingresar claves de las tres geometrías}
\label{\detokenize{nivel_geometrias:ingresar-claves-de-las-tres-geometrias}}
Se declaran en una lista tres claves de las tres geometrías involucradas
en la variable \sphinxstylestrong{clave\_capas}

las claves son de tres caracteres y separadas por comas

\begin{sphinxVerbatim}[commandchars=\\\{\}]
\PYG{n}{clave\PYGZus{}capas}\PYG{o}{=}\PYG{p}{[}\PYG{l+s+s1}{\PYGZsq{}}\PYG{l+s+s1}{mun}\PYG{l+s+s1}{\PYGZsq{}}\PYG{p}{,}\PYG{l+s+s1}{\PYGZsq{}}\PYG{l+s+s1}{anp}\PYG{l+s+s1}{\PYGZsq{}}\PYG{p}{,}\PYG{l+s+s1}{\PYGZsq{}}\PYG{l+s+s1}{usv}\PYG{l+s+s1}{\PYGZsq{}}\PYG{p}{]}
\end{sphinxVerbatim}


\section{Bibliografía}
\label{\detokenize{nivel_geometrias:bibliografia}}

\section{Documentación dentro del código}
\label{\detokenize{nivel_geometrias:documentacion-dentro-del-codigo}}

\chapter{Cobertura de uso y tipo de suelo a nivel municipal}
\label{\detokenize{usvmunicipal:cobertura-de-uso-y-tipo-de-suelo-a-nivel-municipal}}\label{\detokenize{usvmunicipal::doc}}
Objetivo:

Generar bases de datos  a nivel municipal que cuantiquen el área en hectaréas por clase
de uso de suelo y vegetación

\# Insumos
\begin{itemize}
\item {} 
Series I - VI de uso de suelo y vegetacíón INEGI

\item {} 
Municipios del estado de Yucatán (2018)

\end{itemize}

\#\# Procedimiento

Se unificarón las categorias para las seis series publicadas de la siguiente manera:
(Solo aplica para el estado de Yúcatan)

id\_clase - Categoria
1 - Agricultura de riego
2 -Agricultura de temporal
3 - Cuerpo de agua
4 - Manglar
5 - Pastizal
6 - Selva baja
7 - Selva mediana
8 - Sin vegetación
9 - Asentamiento humano
10 - Vegetación de duna costera
11 - Vegetación de petén
12 - Vegetación secundaria de selva baja
13 - Vegetación secundaria de selva mediana
14 - Vegetación secundaria de manglar
15 - Acuícola
16 - Bosque cultivado/Palmar inducido
17 - Tular
18 - Vegetación halófila hidrófila
19 - Sábana

Se genera el script \sphinxstylestrong{datos\_nivel\_municipio.py} el cual genera
realiza los siguientes pasos:
\begin{itemize}
\item {} 
Se declara \sphinxstylestrong{path\_mun} la ruta de la capa de municipios

\item {} 
se realiza un iterador (for) del 1 al 6 para procesar las 6 series de USV

\item {} 
Se declara \sphinxstylestrong{path\_usv} mediante el for la ruta de la capa usv\_serie\_i\_yuc.shp (donde i, va del 1 al 6)

\item {} 
Se declara \sphinxstylestrong{path\_mun\_usv}  mendiante el for la ruta del archivo que resultará de la intersección de municipios y USV  (agregados)

\item {} 
Se declara \sphinxstylestrong{path\_mun\_usv\_csv} mediante el for la ruta del archivo csv que contendrá las áreas (Ha) por clase por municipio

\item {} 
se declara \sphinxstylestrong{path\_mun} como la capa \sphinxstylestrong{municipios}

\item {} 
se declara \sphinxstylestrong{path\_usv} como la cap \sphinxstylestrong{usv}

\item {} 
Se crea una copia de la capa de municipio

\item {} 
Se declara \sphinxstylestrong{path\_interseccion}, mendiante el for que  es la ruta y nombre del resultado de la intersección de municipios y USV

\item {} 
Se crea una lista municipios mendiante el campo \sphinxstylestrong{cve\_mun} de la capa \sphinxstylestrong{municipios}

\item {} 
Se crea una lista de las categorias mediante el campo \sphinxstylestrong{id\_clase} de la capa \sphinxstylestrong{usv}

\item {} 
Se realiza la intersección \sphinxstylestrong{path\_mun\_usv} y \sphinxstylestrong{path\_usv} se indica la ruta y nombre de salida con \sphinxstylestrong{path\_interseccion}

\item {} 
se declara \sphinxstylestrong{path\_mun\_usv} como la capa \sphinxstylestrong{mun\_usv}

\item {} 
Se llama a la funcion \sphinxstylestrong{campos\_clases} pasando como parametros la cap  \sphinxstylestrong{mun\_usv} y la lista\_clases, esta función creará los campos en la capa vectorial

\end{itemize}

como «clase\_i» donde i es el número de id de la clase
- se declara \sphinxstylestrong{path\_interseccion} como la capa \sphinxstylestrong{consulta\_intersect}
- Se inicia la edición de la capa \sphinxstylestrong{mun\_usv}
- se realiza un iterador (for) de la lista\_clases
\begin{itemize}
\item {} 
Se inicializa la variable area = 0

\item {} \begin{description}
\item[{Se realiza un iterador (for) de la lista\_mun}] \leavevmode\begin{itemize}
\item {} 
se restablece la variable area = 0

\item {} 
se realiza un filtro mediante una consulta por municipio y por numero de clase \sphinxstylestrong{request\_mun}

\item {} 
se realiza un filtro mediante una consulta por municipio  \sphinxstylestrong{request\_mun\_o}

\item {} \begin{description}
\item[{Se realiza un iterador (for) de los elementos de la capa \sphinxstylestrong{consulta\_interect} pasando la consulta \sphinxstylestrong{request\_mun}}] \leavevmode\begin{itemize}
\item {} 
Se realiza la suma de los elmentos de la seleccion mediante la funcion geometry().area() y se va guardando en \sphinxstylestrong{area}

\end{itemize}

\end{description}

\item {} \begin{description}
\item[{Se realiza un iterdor (for) de los elementos de la capa \sphinxstylestrong{mun\_usv}  pasando la consulta \sphinxstylestrong{request\_mun}}] \leavevmode\begin{itemize}
\item {} 
Se escribe en el campo \sphinxstylestrong{clase\_i} (donde i es el id de la clase) el valor del área dividido entre 10,000 y redondeado a 2 dígitos

\end{itemize}

\end{description}

\item {} 
Se actualizan los elementos de la capa \sphinxstylestrong{mun\_usv}

\end{itemize}

\end{description}

\end{itemize}
\begin{itemize}
\item {} 
al finalizar la serie de iteradores se guardan los cambios en \sphinxstylestrong{mun\_usv}

\item {} 
Se manda a llamar a la función \sphinxstylestrong{vector\_to\_usv} donde se recibe como parametros la capa \sphinxstylestrong{mun\_usv} y la variable \sphinxstylestrong{path\_mun\_usv\_csv} que es la ruta y el nombre del archivo csv

\end{itemize}

al finalizar se obtiene
\begin{itemize}
\item {} 
6 capas vectoriales a nivel municipal, una por serie \sphinxstylestrong{mun\_usv\_si.shp} (donde i va del 1 al 6)

\item {} 
6 capas vectoriales resultado de las intersecciones \sphinxstylestrong{tp\_inters\_mun\_usv\_si.shp**(donde i va del 1 al 6), una por serie **mun\_usv\_si.shp} donde i va del 1 al 6)

\end{itemize}

se entrega como producto final
\begin{itemize}
\item {} 
6 archivos csv a nivel municipal, uno por serie \sphinxstylestrong{mun\_usv\_si.csv} (donde i va del 1 al 6) que son copia de los atributos de la capa vectorial correspondiente

\end{itemize}

ruta : SIGdesarrollosig\_fomixentregablesmunicipios\_usv


\renewcommand{\indexname}{Índice de Módulos Python}
\begin{sphinxtheindex}
\let\bigletter\sphinxstyleindexlettergroup
\bigletter{a}
\item\relax\sphinxstyleindexentry{apcsig}\sphinxstyleindexpageref{apcsig:\detokenize{module-apcsig}}
\indexspace
\bigletter{i}
\item\relax\sphinxstyleindexentry{indice\_lee\_sallee}\sphinxstyleindexpageref{leesallee:\detokenize{module-indice_lee_sallee}}
\indexspace
\bigletter{s}
\item\relax\sphinxstyleindexentry{sensibilidad\_por\_remocion\_capas}\sphinxstyleindexpageref{analisis:\detokenize{module-sensibilidad_por_remocion_capas}}
\end{sphinxtheindex}

\renewcommand{\indexname}{Índice}
\printindex
\end{document}