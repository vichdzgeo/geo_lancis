%% Generated by Sphinx.
\def\sphinxdocclass{report}
\documentclass[letterpaper,10pt,spanish]{sphinxmanual}
\ifdefined\pdfpxdimen
   \let\sphinxpxdimen\pdfpxdimen\else\newdimen\sphinxpxdimen
\fi \sphinxpxdimen=.75bp\relax

\PassOptionsToPackage{warn}{textcomp}
\usepackage[utf8]{inputenc}
\ifdefined\DeclareUnicodeCharacter
% support both utf8 and utf8x syntaxes
\edef\sphinxdqmaybe{\ifdefined\DeclareUnicodeCharacterAsOptional\string"\fi}
  \DeclareUnicodeCharacter{\sphinxdqmaybe00A0}{\nobreakspace}
  \DeclareUnicodeCharacter{\sphinxdqmaybe2500}{\sphinxunichar{2500}}
  \DeclareUnicodeCharacter{\sphinxdqmaybe2502}{\sphinxunichar{2502}}
  \DeclareUnicodeCharacter{\sphinxdqmaybe2514}{\sphinxunichar{2514}}
  \DeclareUnicodeCharacter{\sphinxdqmaybe251C}{\sphinxunichar{251C}}
  \DeclareUnicodeCharacter{\sphinxdqmaybe2572}{\textbackslash}
\fi
\usepackage{cmap}
\usepackage[T1]{fontenc}
\usepackage{amsmath,amssymb,amstext}
\usepackage{babel}
\usepackage{times}
\usepackage[Sonny]{fncychap}
\ChNameVar{\Large\normalfont\sffamily}
\ChTitleVar{\Large\normalfont\sffamily}
\usepackage{sphinx}

\fvset{fontsize=\small}
\usepackage{geometry}

% Include hyperref last.
\usepackage{hyperref}
% Fix anchor placement for figures with captions.
\usepackage{hypcap}% it must be loaded after hyperref.
% Set up styles of URL: it should be placed after hyperref.
\urlstyle{same}
\addto\captionsspanish{\renewcommand{\contentsname}{Guias de uso:}}

\addto\captionsspanish{\renewcommand{\figurename}{Figura }}
\makeatletter
\def\fnum@figure{\figurename\thefigure{}}
\makeatother
\addto\captionsspanish{\renewcommand{\tablename}{Tabla }}
\makeatletter
\def\fnum@table{\tablename\thetable{}}
\makeatother
\addto\captionsspanish{\renewcommand{\literalblockname}{Lista}}

\addto\captionsspanish{\renewcommand{\literalblockcontinuedname}{proviene de la página anterior}}
\addto\captionsspanish{\renewcommand{\literalblockcontinuesname}{continué en la próxima página}}
\addto\captionsspanish{\renewcommand{\sphinxnonalphabeticalgroupname}{Non-alphabetical}}
\addto\captionsspanish{\renewcommand{\sphinxsymbolsname}{Símbolos}}
\addto\captionsspanish{\renewcommand{\sphinxnumbersname}{Numbers}}

\addto\extrasspanish{\def\pageautorefname{página}}

\setcounter{tocdepth}{0}



\title{Geoprocesamiento en python y qgis Documentation}
\date{02 de marzo de 2020}
\release{1.0}
\author{Víctor Hernández}
\newcommand{\sphinxlogo}{\vbox{}}
\renewcommand{\releasename}{Versión}
\makeindex
\begin{document}

\ifdefined\shorthandoff
  \ifnum\catcode`\=\string=\active\shorthandoff{=}\fi
  \ifnum\catcode`\"=\active\shorthandoff{"}\fi
\fi

\pagestyle{empty}
\sphinxmaketitle
\pagestyle{plain}
\sphinxtableofcontents
\pagestyle{normal}
\phantomsection\label{\detokenize{index::doc}}



\chapter{¿Cómo ejecutar un código en Qgis?}
\label{\detokenize{ejecucion:como-ejecutar-un-codigo-en-qgis}}\label{\detokenize{ejecucion::doc}}

\section{Paso \#1}
\label{\detokenize{ejecucion:paso-1}}
Ejecutar Qgis Desktop 3.XX, la ventana que se muestra es la que
corresponde a la interfaz gráfica del programa, en la barra de tareas
hacer clic en el ícono correspondiente a \sphinxstylestrong{python} para abrir la consola

\noindent\sphinxincludegraphics{{qgis_interfaz}.png}


\section{paso \#2}
\label{\detokenize{ejecucion:paso-2}}
En la parte inferior de la ventana se mostrará la consola de python

\noindent\sphinxincludegraphics{{consola_pyqgis}.png}


\section{paso \#3}
\label{\detokenize{ejecucion:paso-3}}
Hacer clic en el ícono de \sphinxstylestrong{Editor}

\noindent\sphinxincludegraphics{{consola_editor}.png}


\section{paso \#4}
\label{\detokenize{ejecucion:paso-4}}
Se despliega en el lado izquierdo un panel, el cual es el editor
de código, cuenta con una barra de tareas, para abrir un script
dar clic en el ícono de \sphinxstylestrong{abrir archivo}

\noindent\sphinxincludegraphics{{consola_editor_interfaz}.png}


\section{paso \#5}
\label{\detokenize{ejecucion:paso-5}}
Se abrirá una ventana que te permite usar el explorador de archivos
para navegar y encontrar el archivo \sphinxstylestrong{.py}, elegir el script deseado y
dar clic en abrir.

\noindent\sphinxincludegraphics{{abrir_script}.PNG}


\chapter{Análisis de sensibilidad}
\label{\detokenize{analisis:analisis-de-sensibilidad}}\label{\detokenize{analisis::doc}}
Lorem ipsum dolor sit amet, consectetur adipiscing elit.
Aliquam at turpis lacus. Pellentesque vitae efficitur lacus.
Proin eu lectus ultrices mauris viverra vehicula. Proin ante justo,
ultrices eu leo ac, vulputate tristique sapien. Aenean vel enim a elit mollis commodo.
Proin laoreet quis quam quis auctor. Vestibulum nec nisl pretium, bibendum ligula in,
suscipit neque. Nunc placerat ac ipsum vel pellentesque. Phasellus lacinia cursus porttitor.
Donec viverra faucibus nisl, non vestibulum quam posuere sit amet. Nulla a sodales urna.
Donec vestibulum purus purus, iaculis pellentesque eros cursus quis. Maecenas ac maximus sem.

descarga el código de ejemplo

\sphinxcode{\sphinxupquote{sensibilidad.py}}.


\section{Entrada de datos}
\label{\detokenize{analisis:entrada-de-datos}}
El usuario debe reemplazar en el código las variables como pesos \sphinxstylestrong{w},
nombre de los criterios, y  la \sphinxstylestrong{ruta} de donde se encuentran las capas
raster.

ver a partir de la línea 364 en el código de ejemplo

\begin{sphinxVerbatim}[commandchars=\\\{\}]
\PYG{n}{dicc} \PYG{o}{=} \PYG{p}{\PYGZob{}}
    \PYG{l+s+s1}{\PYGZsq{}}\PYG{l+s+s1}{exposicion}\PYG{l+s+s1}{\PYGZsq{}}\PYG{p}{:} \PYG{p}{\PYGZob{}}\PYG{l+s+s1}{\PYGZsq{}}\PYG{l+s+s1}{w}\PYG{l+s+s1}{\PYGZsq{}}\PYG{p}{:}\PYG{l+m+mf}{0.5}\PYG{p}{,}
                        \PYG{l+s+s1}{\PYGZsq{}}\PYG{l+s+s1}{criterios}\PYG{l+s+s1}{\PYGZsq{}}\PYG{p}{:}\PYG{p}{\PYGZob{}}\PYG{l+s+s1}{\PYGZsq{}}\PYG{l+s+s1}{biologico}\PYG{l+s+s1}{\PYGZsq{}}\PYG{p}{:}\PYG{p}{\PYGZob{}}\PYG{l+s+s1}{\PYGZsq{}}\PYG{l+s+s1}{w}\PYG{l+s+s1}{\PYGZsq{}}\PYG{p}{:}\PYG{l+m+mf}{0.50}\PYG{p}{,}
                                                        \PYG{l+s+s1}{\PYGZsq{}}\PYG{l+s+s1}{criterios}\PYG{l+s+s1}{\PYGZsq{}}\PYG{p}{:}\PYG{p}{\PYGZob{}}\PYG{l+s+s1}{\PYGZsq{}}\PYG{l+s+s1}{v\PYGZus{}acuatica}\PYG{l+s+s1}{\PYGZsq{}}\PYG{p}{:}\PYG{p}{\PYGZob{}}\PYG{l+s+s1}{\PYGZsq{}}\PYG{l+s+s1}{w}\PYG{l+s+s1}{\PYGZsq{}}\PYG{p}{:}\PYG{l+m+mf}{0.16}\PYG{p}{,}\PYG{l+s+s1}{\PYGZsq{}}\PYG{l+s+s1}{ruta}\PYG{l+s+s1}{\PYGZsq{}}\PYG{p}{:}\PYG{l+s+s1}{\PYGZsq{}}\PYG{l+s+s1}{\PYGZsq{}}\PYG{p}{\PYGZcb{}}\PYG{p}{,}
                                                                        \PYG{l+s+s1}{\PYGZsq{}}\PYG{l+s+s1}{v\PYGZus{}costera}\PYG{l+s+s1}{\PYGZsq{}}\PYG{p}{:}\PYG{p}{\PYGZob{}} \PYG{l+s+s1}{\PYGZsq{}}\PYG{l+s+s1}{w}\PYG{l+s+s1}{\PYGZsq{}}\PYG{p}{:}\PYG{l+m+mf}{0.84}\PYG{p}{,}\PYG{l+s+s1}{\PYGZsq{}}\PYG{l+s+s1}{ruta}\PYG{l+s+s1}{\PYGZsq{}}\PYG{p}{:}\PYG{l+s+s1}{\PYGZsq{}}\PYG{l+s+s1}{\PYGZsq{}}\PYG{p}{\PYGZcb{}}\PYG{p}{\PYGZcb{}}\PYG{p}{\PYGZcb{}}\PYG{p}{,}
                                        \PYG{l+s+s1}{\PYGZsq{}}\PYG{l+s+s1}{fisico}\PYG{l+s+s1}{\PYGZsq{}}\PYG{p}{:}\PYG{p}{\PYGZob{}}\PYG{l+s+s1}{\PYGZsq{}}\PYG{l+s+s1}{w}\PYG{l+s+s1}{\PYGZsq{}}\PYG{p}{:}\PYG{l+m+mf}{0.50}\PYG{p}{,}
                                                    \PYG{l+s+s1}{\PYGZsq{}}\PYG{l+s+s1}{criterios}\PYG{l+s+s1}{\PYGZsq{}}\PYG{p}{:}\PYG{p}{\PYGZob{}}\PYG{l+s+s1}{\PYGZsq{}}\PYG{l+s+s1}{elevacion}\PYG{l+s+s1}{\PYGZsq{}}\PYG{p}{:}\PYG{p}{\PYGZob{}} \PYG{l+s+s1}{\PYGZsq{}}\PYG{l+s+s1}{w}\PYG{l+s+s1}{\PYGZsq{}}\PYG{p}{:}\PYG{l+m+mf}{0.87}\PYG{p}{,}\PYG{l+s+s1}{\PYGZsq{}}\PYG{l+s+s1}{ruta}\PYG{l+s+s1}{\PYGZsq{}}\PYG{p}{:}\PYG{l+s+s1}{\PYGZsq{}}\PYG{l+s+s1}{\PYGZsq{}}\PYG{p}{\PYGZcb{}}\PYG{p}{,}
                                                                    \PYG{l+s+s1}{\PYGZsq{}}\PYG{l+s+s1}{ancho\PYGZus{}playa}\PYG{l+s+s1}{\PYGZsq{}}\PYG{p}{:}\PYG{p}{\PYGZob{}}\PYG{l+s+s1}{\PYGZsq{}}\PYG{l+s+s1}{w}\PYG{l+s+s1}{\PYGZsq{}}\PYG{p}{:}\PYG{l+m+mf}{0.13}\PYG{p}{,}\PYG{l+s+s1}{\PYGZsq{}}\PYG{l+s+s1}{ruta}\PYG{l+s+s1}{\PYGZsq{}}\PYG{p}{:}\PYG{l+s+s1}{\PYGZsq{}}\PYG{l+s+s1}{\PYGZsq{}}\PYG{p}{\PYGZcb{}}
    \PYG{p}{\PYGZcb{}}\PYG{p}{\PYGZcb{}}\PYG{p}{\PYGZcb{}}\PYG{p}{\PYGZcb{}}\PYG{p}{,}
    \PYG{l+s+s1}{\PYGZsq{}}\PYG{l+s+s1}{susceptibilidad}\PYG{l+s+s1}{\PYGZsq{}}\PYG{p}{:} \PYG{p}{\PYGZob{}}\PYG{l+s+s1}{\PYGZsq{}}\PYG{l+s+s1}{w}\PYG{l+s+s1}{\PYGZsq{}}\PYG{p}{:}\PYG{l+m+mf}{0.5}\PYG{p}{,}
                            \PYG{l+s+s1}{\PYGZsq{}}\PYG{l+s+s1}{criterios}\PYG{l+s+s1}{\PYGZsq{}}\PYG{p}{:}\PYG{p}{\PYGZob{}}\PYG{l+s+s1}{\PYGZsq{}}\PYG{l+s+s1}{biologico}\PYG{l+s+s1}{\PYGZsq{}}\PYG{p}{:}\PYG{p}{\PYGZob{}}\PYG{l+s+s1}{\PYGZsq{}}\PYG{l+s+s1}{w}\PYG{l+s+s1}{\PYGZsq{}}\PYG{p}{:}\PYG{l+m+mf}{0.50} \PYG{p}{,}
                                                            \PYG{l+s+s1}{\PYGZsq{}}\PYG{l+s+s1}{criterios}\PYG{l+s+s1}{\PYGZsq{}}\PYG{p}{:}\PYG{p}{\PYGZob{}}\PYG{l+s+s1}{\PYGZsq{}}\PYG{l+s+s1}{v\PYGZus{}costera}\PYG{l+s+s1}{\PYGZsq{}}\PYG{p}{:}\PYG{p}{\PYGZob{}} \PYG{l+s+s1}{\PYGZsq{}}\PYG{l+s+s1}{w}\PYG{l+s+s1}{\PYGZsq{}}\PYG{p}{:}\PYG{l+m+mf}{1.0}\PYG{p}{,}\PYG{l+s+s1}{\PYGZsq{}}\PYG{l+s+s1}{ruta}\PYG{l+s+s1}{\PYGZsq{}}\PYG{p}{:}\PYG{l+s+s1}{\PYGZsq{}}\PYG{l+s+s1}{\PYGZsq{}}\PYG{p}{\PYGZcb{}}\PYG{p}{\PYGZcb{}}\PYG{p}{\PYGZcb{}}\PYG{p}{,}
                                            \PYG{l+s+s1}{\PYGZsq{}}\PYG{l+s+s1}{fisico}\PYG{l+s+s1}{\PYGZsq{}}\PYG{p}{:}\PYG{p}{\PYGZob{}}\PYG{l+s+s1}{\PYGZsq{}}\PYG{l+s+s1}{w}\PYG{l+s+s1}{\PYGZsq{}}\PYG{p}{:}\PYG{l+m+mf}{0.50}\PYG{p}{,}
                                                        \PYG{l+s+s1}{\PYGZsq{}}\PYG{l+s+s1}{criterios}\PYG{l+s+s1}{\PYGZsq{}}\PYG{p}{:}\PYG{p}{\PYGZob{}}\PYG{l+s+s1}{\PYGZsq{}}\PYG{l+s+s1}{elevacion}\PYG{l+s+s1}{\PYGZsq{}}\PYG{p}{:}\PYG{p}{\PYGZob{}} \PYG{l+s+s1}{\PYGZsq{}}\PYG{l+s+s1}{w}\PYG{l+s+s1}{\PYGZsq{}}\PYG{p}{:}\PYG{l+m+mf}{0.26}\PYG{p}{,}\PYG{l+s+s1}{\PYGZsq{}}\PYG{l+s+s1}{\PYGZsq{}}\PYG{n}{ruta}\PYG{l+s+s1}{\PYGZsq{}}\PYG{l+s+s1}{:}\PYG{l+s+s1}{\PYGZsq{}}\PYG{l+s+s1}{\PYGZsq{}}\PYG{l+s+s1}{ \PYGZcb{},}
                                                                        \PYG{l+s+s1}{\PYGZsq{}}\PYG{l+s+s1}{duna\PYGZus{}costera}\PYG{l+s+s1}{\PYGZsq{}}\PYG{p}{:}\PYG{p}{\PYGZob{}}\PYG{l+s+s1}{\PYGZsq{}}\PYG{l+s+s1}{w}\PYG{l+s+s1}{\PYGZsq{}}\PYG{p}{:}\PYG{l+m+mf}{0.10}\PYG{p}{,}\PYG{l+s+s1}{\PYGZsq{}}\PYG{l+s+s1}{ruta}\PYG{l+s+s1}{\PYGZsq{}}\PYG{p}{:}\PYG{l+s+s1}{\PYGZsq{}}\PYG{l+s+s1}{\PYGZsq{}}\PYG{p}{\PYGZcb{}}\PYG{p}{,}
                                                                        \PYG{l+s+s1}{\PYGZsq{}}\PYG{l+s+s1}{tipo\PYGZus{}litoral}\PYG{l+s+s1}{\PYGZsq{}}\PYG{p}{:}\PYG{p}{\PYGZob{}}\PYG{l+s+s1}{\PYGZsq{}}\PYG{l+s+s1}{w}\PYG{l+s+s1}{\PYGZsq{}}\PYG{p}{:}\PYG{l+m+mf}{0.64}\PYG{p}{,}\PYG{l+s+s1}{\PYGZsq{}}\PYG{l+s+s1}{ruta}\PYG{l+s+s1}{\PYGZsq{}}\PYG{p}{:}\PYG{l+s+s1}{\PYGZsq{}}\PYG{l+s+s1}{\PYGZsq{}}\PYG{p}{\PYGZcb{}}\PYG{p}{,}
    \PYG{p}{\PYGZcb{}}\PYG{p}{\PYGZcb{}}\PYG{p}{\PYGZcb{}}\PYG{p}{\PYGZcb{}}

    \PYG{p}{\PYGZcb{}}
\end{sphinxVerbatim}


\section{Salida de datos}
\label{\detokenize{analisis:salida-de-datos}}
la variable \sphinxstylestrong{p\_procesamiento} indica la ruta donde se escribirán las
capas integradas y el archivo csv que contendrá el análisis de sensibilidad


\section{Ejemplo}
\label{\detokenize{analisis:ejemplo}}

\subsection{Insumos}
\label{\detokenize{analisis:insumos}}
Descarga los insumos para este ejemplo  \sphinxcode{\sphinxupquote{aqui}}


\subsection{Procedimiento}
\label{\detokenize{analisis:procedimiento}}
Abre el código \sphinxstylestrong{sensibilidad.py} en Qgis 3.14 o superior,
Si tienes dudas de como hacerlo visualiza la \sphinxhref{https://vichdzgeo.github.io/geo\_lancis/ejecucion.html}{guia}

Modificar las rutas donde se encuentran los insumos y
elegir una carpeta en donde se escribiran los resultados

\noindent\sphinxincludegraphics{{modificar_paths}.PNG}

El tiempo de ejecución del código en este ejemplo es de 10 minutos.
al finalizar se mostrará la consola de la siguiente manera:

\noindent\sphinxincludegraphics{{fin_ejecucion}.PNG}

el archivo csv de salida que contiene los datos es el siguiente:


\begin{savenotes}\sphinxattablestart
\centering
\sphinxcapstartof{table}
\sphinxthecaptionisattop
\sphinxcaption{Analisis de sensibilidad}\label{\detokenize{analisis:id1}}
\sphinxaftertopcaption
\begin{tabulary}{\linewidth}[t]{|T|T|T|T|T|T|T|T|T|}
\hline
\sphinxstyletheadfamily 
criterio
&\sphinxstyletheadfamily 
exp\_media
&\sphinxstyletheadfamily 
sensibilidad\_exp
&\sphinxstyletheadfamily 
sus\_media
&\sphinxstyletheadfamily 
sensibilidad\_sus
&\sphinxstyletheadfamily 
res\_media
&\sphinxstyletheadfamily 
sensibilidad\_res
&\sphinxstyletheadfamily 
vulnerabilidad
&\sphinxstyletheadfamily 
sensibilidad\_vul
\\
\hline
total
&
0.79
&&
0.21
&&
0.22
&&
0.79
&\\
\hline
v\_acuatica
&
0.77
&
0.03
&
0.21
&
0.00
&
0.22
&
0.00
&
0.77
&
0.03
\\
\hline
v\_costera
&
0.91
&
0.15
&
0.28
&
0.33
&
0.22
&
0.00
&
0.91
&
0.15
\\
\hline
elevacion
&
0.76
&
0.04
&
0.26
&
0.20
&
0.22
&
0.03
&
0.77
&
0.02
\\
\hline
ancho\_playa
&
0.80
&
0.01
&
0.21
&
0.00
&
0.27
&
0.18
&
0.79
&
0.00
\\
\hline
v\_costera
&
0.91
&
0.15
&
0.28
&
0.33
&
0.22
&
0.00
&
0.91
&
0.15
\\
\hline
elevacion
&
0.76
&
0.04
&
0.26
&
0.20
&
0.22
&
0.03
&
0.77
&
0.02
\\
\hline
duna\_costera
&
0.79
&
0.00
&
0.23
&
0.07
&
0.23
&
0.04
&
0.79
&
0.00
\\
\hline
tipo\_litoral
&
0.79
&
0.00
&
0.08
&
0.60
&
0.21
&
0.08
&
0.77
&
0.03
\\
\hline
biodiversidad
&
0.79
&
0.00
&
0.21
&
0.00
&
0.22
&
0.00
&
0.79
&
0.00
\\
\hline
servicios\_ambientales
&
0.79
&
0.00
&
0.21
&
0.00
&
0.22
&
0.00
&
0.79
&
0.00
\\
\hline
ancho\_playa
&
0.80
&
0.01
&
0.21
&
0.00
&
0.27
&
0.18
&
0.79
&
0.00
\\
\hline
duna\_costera
&
0.79
&
0.00
&
0.23
&
0.07
&
0.23
&
0.04
&
0.79
&
0.00
\\
\hline
elevacion
&
0.76
&
0.04
&
0.26
&
0.20
&
0.22
&
0.03
&
0.77
&
0.02
\\
\hline
tipo\_litoral
&
0.79
&
0.00
&
0.08
&
0.60
&
0.21
&
0.08
&
0.77
&
0.03
\\
\hline
\end{tabulary}
\par
\sphinxattableend\end{savenotes}


\section{Documentación del código}
\label{\detokenize{analisis:documentacion-del-codigo}}

\chapter{Índice Lee-Sallee}
\label{\detokenize{leesallee:indice-lee-sallee}}\label{\detokenize{leesallee::doc}}
Lorem ipsum dolor sit amet, consectetur adipiscing elit.
Aliquam at turpis lacus. Pellentesque vitae efficitur lacus.
Proin eu lectus ultrices mauris viverra vehicula. Proin ante justo,
ultrices eu leo ac, vulputate tristique sapien. Aenean vel enim a elit mollis commodo.
\begin{equation*}
\begin{split}lee\_sallee\_index =  \frac{A\cap B}{A\cup B}\end{split}
\end{equation*}
descarga el código de ejemplo

\sphinxcode{\sphinxupquote{indice\_lee\_sallee.py}}.


\section{Ejemplo}
\label{\detokenize{leesallee:ejemplo}}

\subsection{Insumos}
\label{\detokenize{leesallee:insumos}}
Descarga los insumos para este ejemplo \sphinxcode{\sphinxupquote{aqui}}


\subsection{Procedimiento}
\label{\detokenize{leesallee:procedimiento}}
Abre el código \sphinxstylestrong{indice\_lee\_sallee.py} en Qgis 3.14 o superior,
Si tienes dudas de como hacerlo visualiza la \sphinxhref{https://vichdzgeo.github.io/geo\_lancis/ejecucion.html}{guia}

Modifica las rutas donde se encuentran los insumos
\begin{itemize}
\item {} 
vlayer\_base es la capa vectorial que se ocupa como base

\item {} 
vlayer\_model es la capa vectorial que se ocupa como modelo

\end{itemize}

\noindent\sphinxincludegraphics{{rutas}.png}
\phantomsection\label{\detokenize{leesallee:module-indice_lee_sallee}}\index{indice\_lee\_sallee (módulo)@\spxentry{indice\_lee\_sallee}\spxextra{módulo}}\index{indice\_lee\_salee() (en el módulo indice\_lee\_sallee)@\spxentry{indice\_lee\_salee()}\spxextra{en el módulo indice\_lee\_sallee}}

\begin{fulllineitems}
\phantomsection\label{\detokenize{leesallee:indice_lee_sallee.indice_lee_salee}}\pysiglinewithargsret{\sphinxcode{\sphinxupquote{indice\_lee\_sallee.}}\sphinxbfcode{\sphinxupquote{indice\_lee\_salee}}}{\emph{vlayer\_base}, \emph{vlayer\_model}, \emph{campo\_categoria}, \emph{categoria}, \emph{id='ageb\_id'}}{}~\begin{quote}

Esta función regresa el índice Lee-Sallee
\begin{equation*}
\begin{split}lee\_sallee\_index =  \end{split}
\end{equation*}\end{quote}

rac\{Acap B\}\{Acup B\}
\begin{quote}
\begin{quote}\begin{description}
\item[{param vlayer\_base}] \leavevmode
vector base

\item[{type vlayer\_base}] \leavevmode
QgsVectorLayer

\item[{param vlayer\_model}] \leavevmode
Vector modelo

\item[{type vlayer\_model}] \leavevmode
QgsVectorLayer

\item[{param campo\_categoria}] \leavevmode
Nombre del campo que tiene las categorias

\item[{type campo\_categoria}] \leavevmode
String

\item[{param categoria}] \leavevmode
Número de categoria

\item[{type categoria}] \leavevmode
int

\item[{param id}] \leavevmode
nombre del campo identificador

\item[{type id}] \leavevmode
String

\end{description}\end{quote}
\end{quote}

\end{fulllineitems}

\index{lista\_shp() (en el módulo indice\_lee\_sallee)@\spxentry{lista\_shp()}\spxextra{en el módulo indice\_lee\_sallee}}

\begin{fulllineitems}
\phantomsection\label{\detokenize{leesallee:indice_lee_sallee.lista_shp}}\pysiglinewithargsret{\sphinxcode{\sphinxupquote{indice\_lee\_sallee.}}\sphinxbfcode{\sphinxupquote{lista\_shp}}}{\emph{path\_carpeta}}{}~\begin{quote}\begin{description}
\item[{Parámetros}] \leavevmode
\sphinxstyleliteralstrong{\sphinxupquote{path\_carpeta}} (\sphinxstyleliteralemphasis{\sphinxupquote{String}}) \textendash{} ruta que contiene los archivos shape a procesar

\end{description}\end{quote}

\end{fulllineitems}



\renewcommand{\indexname}{Índice de Módulos Python}
\begin{sphinxtheindex}
\let\bigletter\sphinxstyleindexlettergroup
\bigletter{i}
\item\relax\sphinxstyleindexentry{indice\_lee\_sallee}\sphinxstyleindexpageref{leesallee:\detokenize{module-indice_lee_sallee}}
\end{sphinxtheindex}

\renewcommand{\indexname}{Índice}
\printindex
\end{document}